\documentclass[11pt]{book}
\pagestyle{plain}
%\documentclass{article}

\usepackage[utf8]{inputenc}
\usepackage[english]{babel}
\usepackage[T1]{fontenc}
\usepackage{latexsym,amsmath,amssymb}
\usepackage{amsthm}
\usepackage{amsfonts}
\usepackage{geometry}
\usepackage{graphicx}
\usepackage{lmodern}
\usepackage{pifont}
\usepackage{tikz}
\usepackage{pgfplots}
\usepackage{thmtools}
\usepackage{wrapfig}
\usepackage{extarrows}
\usepackage{breqn}
\usepackage{physics}
\usepackage{afterpage}
\usepackage[inline]{enumitem}
\usepackage{mathrsfs}
\usepackage{scalerel}
\usepackage{stackengine,wasysym}
\usepackage{aligned-overset}
\usepackage{stackengine}
\usepackage{mathtools}
\usepackage{nccmath}
\usepackage{url}
\usepackage{float}
\usepackage{lipsum}
\usepackage[toc]{appendix}
\usepackage{chngcntr}
\usepackage{etoolbox}
\usepackage{framed}
\usepackage{mdframed}
\usepackage{blindtext}
\usepackage{xcolor}
\usepackage{fancyhdr}
\usepackage{titlesec}
\usepackage{esint}
\usepackage{aligned-overset}
\usepackage{mathrsfs}
\graphicspath{{images/}}

\DeclarePairedDelimiter\ceil{\lceil}{\rceil}
\DeclarePairedDelimiter\floor{\lfloor}{\rfloor}


\titleformat{\chapter}[block]{\huge\bfseries\itshape\raggedleft}{\chaptertitlename\  \thechapter.}{0.5ex}{}[]

\titleformat{\section}[block]{\Large\bfseries\itshape}{\thesection\ }{0.5ex}{}[]

\titleformat{\subsection}[block]{\large\bfseries\itshape}{\thesubsection\ }{0.5ex}{}[]

\usepackage{hyperref}
\hypersetup{
    colorlinks = true,
    linkcolor = blue,
    filecolor = blue,      
    urlcolor = blue,
    citecolor = blue,
    pdftitle = {Sharelatex Example},
    bookmarks = true,
    pdfpagemode = FullScreen,
}

\urlstyle{same}

\newcommand*\circled[1]{\tikz[baseline=(char.base)]{
            \node[shape=circle,draw,inner sep=2pt] (char) {#1};}}

\setlength{\oddsidemargin}{1pt}
\setlength{\evensidemargin}{1pt}
\setlength{\marginparwidth}{30pt} % these gain 53pt width
\setlength{\topmargin}{1pt}       % gains 26pt height
\setlength{\headheight}{1pt}      % gains 11pt height
\setlength{\headsep}{1pt}         % gains 24pt height
%\setlength{\footheight}{12 pt} 	  % cannot be changed as number must fit
\setlength{\footskip}{24pt}       % gains 6pt height
\setlength{\textheight}{650pt}    % 528 + 26 + 11 + 24 + 6 + 55 for luck
\setlength{\textwidth}{460pt}     % 360 + 53 + 47 for luck

\title{Sections and Chapters}

\newtheorem{definition}{Definition}[chapter]
\newtheorem{theorem}{Theorem}[chapter]
\newtheorem{corollary}{Corollary}[theorem]
\newtheorem{lemma}{Lemma}[chapter]
\newtheorem{proposition}{Proposition}[chapter]
\newtheorem{exercise}{Exercise}[section]
\newtheorem{remark}{Remark}[chapter]
\theoremstyle{definition}
\newtheorem{example}{Example}[chapter]
\numberwithin{equation}{chapter}


\AtBeginEnvironment{subappendices}{%
\Appendix*{Appendix}
\addcontentsline{toc}{chapter}{Appendices}
\counterwithin{figure}{section}
\counterwithin{table}{section}
}

\newmdenv[
  linewidth=1.5pt, 
  topline=false, 
  bottomline=false, 
  rightline=false,
  innerleftmargin=15pt,
  leftmargin=10pt,
  rightmargin=0pt,
  innerrightmargin=0pt, 
]{claim}

\def\MM{\mathfrak{M}}
\def\BB{\mathfrak{B}}
\def\CC{\mathfrak{C}}
\def\leb{{\mathcal L}}
\def\H{{\mathcal H}}
\def\L{{\mathcal L}}
\def\diam{{\operatorname{diam}\,}}
\def\co{{\overline{\operatorname{co}}\,}}
\def\dist{{\operatorname{dist}\,}}

\usepackage{scalerel}
%\usepackage[usestackEOL]{stackengine}
\def\avint{\,\ThisStyle{\ensurestackMath{%
  \stackinset{c}{.2\LMpt}{c}{1\LMpt}{\SavedStyle-}{\SavedStyle\phantom{\int}}}%
  \setbox0=\hbox{$\SavedStyle\int\,$}\kern-\wd0}\int}

\pagestyle{fancy}
\fancyhf{}
\fancyhead[LE]{{\em \thepage}}
\fancyhead[RE]{{\em \leftmark}}
\fancyhead[LO]{{\em \rightmark}}
\fancyhead[RO]{{\em \thepage}}


\counterwithout{footnote}{chapter}

\def\dsp{\def\baselinestretch{1.35}\large
\normalsize}
%%%%This makes a double spacing. Use this with 11pt style. If you
%%%%want to use this just insert \dsp after the \begin{document}
%%%%The correct baselinestretch for double spacing is 1.37. However
%%%%you can use different parameter.

\newcommand\blankpage{%
    \null
    \thispagestyle{empty}%
    \addtocounter{page}{-1}%
    \newpage}
    
\def\U{{\mathcal U}}

\begin{document}
\frontmatter

\begin{titlepage}
	\begin{center}
	\textbf{\LARGE{}} \\
	\vspace{40mm}
    \textbf{\Huge{Mathematical Analysis}} \\
    \medskip
    \vspace{10mm} %5mm vertical space
    \large{\textsc{Zhen Yao}}\\
    %\large{\textsc{University of Pittsburgh}}
    \end{center}
\end{titlepage}

\tableofcontents{}
\mainmatter

\newpage

\chapter*{Preface}
\addcontentsline{toc}{chapter}{Preface}

The main purpose of this book is to help prepare for the preliminary exam in Mathematical Analysis as a new mathematics PhD student. This book will follow the structure in Rudin's {\it Principles of Mathematical Analysis}\cite{1} and more contents will be added in each chapters, some are from Dr. Piotr Hajłasz's {\it Introduction to Analysis}\cite{2}.

The content of real and complex number system, especially how to constructing the real number system, will not be covered in this book, since it is perfectly discussed in both Rudin's and Dr. Hajłasz's books. Thus, this book starts with the basic topology. 

\null\hfill ZHEN YAO

\newpage

\chapter{Basic Topology}

\section{Functions}

\begin{definition}
Let $X$ and $Y$ be two sets, and suppose with each element $x$ of $X$ there is associated an element of $Y$, denoted by $f(x)$. Then $f$ is said to to a function from $X$ to $Y$ (or a mapping of $X$ into $Y$), and we write it as $f: X \to Y$. Also,
\begin{enumerate}[label=(\alph*)]
    \item the set $X$ is called the domain of $f$, and $f$ is said to be defined on $X$,
    
    \item the set $Y$ is called the target of $f$,
    
    \item the set of all values of $f(X) \coloneqq \{y \in Y \,:\, \exists x \in X, y = f(x)\}$ is called the range of $f$. If $E \subset X$, we call $f(E)$ the image of $E$ under $f$, and clearly $f(E) \subset Y$.
    
    \item the graph of $f$ is a subset of $X \times Y$, defined by
    \begin{align*}
        \operatorname{graph}(f) \coloneqq \left\{(x,f(x)) \in X \times Y \,:\, x \in X \right\}.
    \end{align*}
\end{enumerate}
\end{definition}

\medskip

\begin{definition}
Let $f: X \to Y$. If $f(X) = Y$, we say $f$ is onto or surjective if $f(X) = Y$. Clearly, the term of onto is more specific than into.

If $E \subset Y$, $f^{-1}(E)$ denotes the set of all $x \in X$ such that $f(x) \in E$. We call $f^{-1}(E)$ the inverse image of $E$ under $f$. If for each $y \in Y$, $f^{-1}(y)$ consists of at most one element of $X$, then $f$ is said to be one-to-one or injective, i.e. if for any $x_1 \neq x_2, x_1, x_2 \in X$, then $f(x_1) \neq f(x_2)$.
\end{definition}

\medskip

\begin{definition}
A function $f:X \to Y$ that is one-to-one and onto is called a bijection. In this case, the inverse function is defined on $Y = f(X)$ and the inverse function $f^{-1}: Y \to X$ is also a bijection.
\end{definition}

\begin{definition}
If $f: X \to Y$ and $g: Y \to Z$ are two functions, then the composition of $f$ and $g$ is the function $g \circ f: X \to Z$ defined by $(g \circ f)(x) = g(f(x))$ for $x \in X$.
\end{definition}

\medskip

\begin{definition}
If $f: X \to Y$ and $A \subset X$, then the restriction of $f$ to $A$ is $f|_A: A \to Y$ defined by $f|_A(x) = f(x)$ for all $x \in A$. In this case, we say that $f$ is an extension of $f|_A$.
\end{definition}

\medskip

\begin{proposition}
Let $f: X \to Y$ be a function and let $A, B \subset Y$. Then,
\begin{align*}
    f^{-1}(A \cup B) & = f^{-1}(A) \cup f^{-1}(B), \\
    f^{-1}(A \cap B) & = f^{-1}(A) \cap f^{-1}(B), \\
    f^{-1}(Y \setminus A) & = X \setminus f^{-1}(A).
\end{align*}
\end{proposition}

\begin{remark}
Given a set $E \subset X$, $E^c = X \setminus E$ denotes the complement of the set $E$ in $X$. With this notation, the last equality can be expressed as
\begin{align*}
    f^{-1}(A^c) = \left(f^{-1}(A)\right)^c.
\end{align*}
\end{remark}
\begin{proof}
First, if $x \in f^{-1}(A \cup B)$, then $f(x) \in A \cup B$, and hence $f(x) \in A$ or $f(x) \in B$. If $f(x) \in A$, then clearly $x \in f^{-1}(A) \subset f^{-1}(A) \cup f^{-1}(B)$. If $f(x) \in B$, then clearly $x \in f^{-1}(B) \subset f^{-1}(A) \cup f^{-1}(B)$. And hence $f^{-1}(A \cup B) \subset f^{-1}(A) \cup f^{-1}(B)$. On the other hand, if $x \in f^{-1}(A) \cup f^{-1}(B)$, then $x \in f^{-1}(A)$ or $x \in f^{-1}(B)$. If $x \in f^{-1}(A)$, then $f(x) \in A \subset A \cup B$. If $x \in f^{-1}(B)$, then $f(x) \in B \subset A \cup B$. In either case $f(x) \in A \cup B$, and hence $x \in f^{-1}(A \cup B)$. This completes the proof of the first equality.

Second, if $x \in f^{-1}(A \cap B)$, then $f(x) \in A \cap B$, hence $f(x) \in A$ and $f(X) \in B$. In this case, $x \in f^{-1}(A) \cap f^{-1}(B)$. On the other hand, if $x \in f^{-1}(A) \cap f^{-1}(B)$, then $f(x) \in A$ and $f(x) \in B$. Hence, $f(x) \in A \cap B$ and this implies $x \in f^{-1}(A \cap B)$. This completes the proof of the second equality.

Finally, if $x \in f^{-1}(Y \setminus A)$, then $f(x) \in Y \setminus A$, and hence $x \notin f^{-1}(A)$. This implies $x \in X \setminus f^{-1}(A)$. On the other hand, if $x \in X \setminus f^{-1}(A)$, then $f(x) \notin A$, and hence $f(x) \in Y \setminus Y$, and this implies $x \in f^{-1}(Y \setminus A)$. 
\end{proof}

\medskip

\begin{proposition}
Let $f: X \to Y$ be a function and let $A, B \subset X$. Then,
\begin{align*}
    f(A \cup B) = f(A) \cup f(B), \qquad f(A \cap B) \subset f(A) \cap f(B).
\end{align*}
If in addition $f$ is one-to-one then
\begin{align*}
    f(A \cap B) = f(A) \cap f(B).
\end{align*}
\end{proposition}
\begin{proof}
First, if $y \in f(A \cup B)$, then $y = f(x)$ for some $x \in A \cup B$. If $x \in A$, then $y = f(x) \in f(A)$. If $x \in B$, then $y = f(x) \in f(B)$. In either case, we have $y \in f(A) \cup f(B)$. On the other hand, if $y \in f(A) \cup f(B)$, then $y \in f(A)$ or $y \in f(B)$. If $y \in f(A)$, then $y = f(x)$ for some $x \in A$. If $y \in f(B)$, then $y = f(x)$ for some $x \in B$. In either case, we have $y = f(x) \in f(A \cup B)$.

Second, if $y \in f(A \cap B)$, then $y = f(x)$ for some $x \in A \cap B$. Then, $y = f(x) \in f(A)$ and $y = f(x) \in f(B)$, hence $y \in f(A) \cap f(B)$. This implies $f(A \cap B) \subset f(A) \cap f(B)$.

Finally, if $f$ is one-to-one, it remains to show that $f(A) \cap f(B) \subset f(A \cap B)$. If $y \in f(A) \cap f(B)$, then there exist $x_1 \in A$ and $x_2 \in B$ such that $y = f(x_1)$ and $y = f(x_2)$. Since $f$ is one-to-one, then $x_1 = x_2$ and clearly $x_1 = x_2 \in A \cap B$, and hence $y = f(x_1) = f(x_2) \in f(A \cap B)$. This completes the proof of $f(A) \cap f(B) \subset f(A \cap B)$.
\end{proof}

\begin{remark}
Why $f(A) \cap f(B) \subset f(A \cap B)$ fail to hold when $f$ is not one-to-one? If $y \in f(A) \cap f(B)$, then $y = f(x_1)$ for some $x_1 \in A$ and $y = f(x_2)$ for some $x_2 \in B$. However, $x_1$ may not be equal to $x_2$ and hence we cannot claim there is a common $x \in A \cap B$ such that $y = f(x)$. And this proposition leads to the following results.
\end{remark}

\medskip

\begin{proposition}
Let $f: X \to Y$ be a function and $A_1, A_2, A_3, \cdots$ are subsets of $X$, then
\begin{align*}
    f \left(\bigcup^\infty_{i=1} A_i\right) = \bigcup^\infty_{i=1} f(A_i), \qquad f \left(\bigcap^\infty_{i=1} A_i\right) \subset \bigcup^\infty_{i=1} f(A_i).
\end{align*}
If in addition $f$ is one-to-one, then
\begin{align*}
    f \left(\bigcap^\infty_{i=1} A_i\right) = \bigcup^\infty_{i=1} f(A_i).
\end{align*}
\end{proposition}


\medskip

\section{Finite, Countable, and Uncountable Sets}

\begin{definition}
If there is a one-to-one mapping of $X$ onto $Y$, we say that $X$ and $Y$ is one-to-one correspondence, or that $X$ and $Y$ have the same cardinal number, or $X$ and $Y$ are equivalent, and we write $X \sim Y$. Clearly, this relation has the following properties:
\begin{enumerate}[label=(\alph*)]
    \item It is reflexive: $X \sim X$.
    
    \item It is symmetric: If $X \sim Y$, then $Y \sim X$.
    
    \item It is transitive: If $X \sim Y$ and $Y \sim Z$, then $X \sim Z$.
\end{enumerate}
Any relation with these three properties is called an equivalence relation.
\end{definition}

\medskip

\begin{remark}
A bijection $f:X \to Y$ is a one-to-one correspondence between $X$ and $Y$.
\end{remark}

\medskip

\begin{definition}
For any positive integer $n \in \mathbb{N}$, where $\mathbb{N}$ is the set of all positive integers, let $J_n = \{1,2,\cdots,n\}$. For any set $A$, we say:
\begin{enumerate}[label=(\alph*)]
    \item $A$ is finite if $A \sim J_n$ for some $n$ (the empty set $\emptyset$ is also considered to be finite).
    
    \item $A$ is infinite if $A$ is not finite.
    
    \item $A$ is countable if $A \sim \mathbb{N}$ (Countable sets are sometimes called enumerable or denumerable).
    
    \item $A$ is at most countable if $A$ is finite or countable.
\end{enumerate}
\end{definition}

\medskip

\begin{proposition}
The sets $\mathbb{N}$ and $2\mathbb{N} \coloneqq \{2n \,:\, x \in \mathbb{N}\}$ have the same cardinality.
\end{proposition}
\begin{proof}
Indeed, let $f(x) = 2n$ for $n \in \mathbb{N}$, then $f: \mathbb{N} \to 2\mathbb{N}$ is clearly a bijection.
\end{proof}

\medskip

\begin{proposition}
The sets $\mathbb{N}$ and $\mathbb{Z}$ have the same cardinality.
\end{proposition}
\begin{proof}
Indeed, for $n \in \mathbb{N}$, let
\begin{align*}
    f(n) = \begin{cases}
        \frac{n}{2}, & n \,\,\text{even}, \\
        - \frac{n-1}{2}, & n \,\,\text{odd}.
    \end{cases}
\end{align*}
Clearly, $f: \mathbb{N} \to \mathbb{Z}$ is clearly a bijection.
\end{proof}

\medskip

\begin{definition}
A sequence is a function $f$ defined on the set $\mathbb{N}$. If $f(n) = x_n$, for $n \in \mathbb{N}$, denote the sequence $f$ by ${x_n}$, or sometimes by $\{x_1, x_2, x_3, \cdots\}$. The values of $f$, that is, $x_n$, are called the terms of the sequence. If $A$ is a set and if $x_n \in A$ for all $n \in \mathbb{N}$, then $\{x_n\}$ is said to be a sequence in $A$, or a sequence of elements of $A$.

Since every countable set is the range of a one-to-one function defined on $\mathbb{N}$, we may regard every countable set as the range of a sequence of distinct terms.
\end{definition}

\medskip

\begin{theorem}\label{th_11}
Every infinite subset of a countable set $A$ is countable.
\end{theorem}
\begin{proof}
Suppose $E \subset A$, and $E$ is infinite. Arrange the elements of $A$ in a sequence $\{x_n\}$. Now, let $n_{1}$ be the smallest integer such that $x_{n_1} \in E$. After choosing $n_1, \cdots, n_{k-1}$, let $n_k$ be the smallest integer greater than $n_{k-1}$ such that $x_{n_k} \in E$. Let $f(k) = x_{n_k}$ for $k \in \mathbb{N}$, we have a one-to-one correspondence between $E$ and $\mathbb{N}$.
\end{proof}

\medskip

\begin{theorem}\label{th_12}
Let $\{E_n\}, n \in \mathbb{N}$ be a sequence of countable sequences and let 
\begin{align}\label{th_11_equ_1}
    S = \bigcup^\infty_{n=1} E_n.
\end{align}
Then $S$ is countable.
\end{theorem}
\begin{proof}
Let every set $E_n$ be arranged in a sequence $\{x_{nk}\}, k \in \mathbb{N}$, and consider the infinite array:
\begin{align*}
    x_{11}, \quad x_{12}, \quad x_{13}, \quad x_{14}, \quad \cdots \\
    x_{21}, \quad x_{22}, \quad x_{23}, \quad x_{24}, \quad \cdots \\
    x_{31}, \quad x_{32}, \quad x_{33}, \quad x_{34}, \quad \cdots
\end{align*}
where the elements of $E_n$ form the $n$th row. These elements can be arranged into a sequence
\begin{align*}
    x_{11}, x_{21}, x_{12}, x_{31}, x_{22}, x_{13}, x_{41}, x_{32}, x_{23}, x_{14}, \cdots
\end{align*}
If any two of the set $E_n$ have elements in common, then there is a subset $T$ of $\mathbb{N}$ such that $S \sim T$, hence by Theorem \ref{th_11}, $S$ is at most countable. Also, $E_1 \subset S$ is infinite, then $S$ is also infinite, and thus countable.
\end{proof}

\medskip

\begin{corollary}
Suppose $A$ is at most countable, and for every $\alpha \in A$, $E_{\alpha}$ is at most countable. Then,
\begin{align*}
    T = \bigcup_{\alpha \in A} E_{\alpha}
\end{align*}
is at most countable.
\end{corollary}
\begin{proof}
For $T$ is equivalent to a subset of \eqref{th_11_equ_1}.
\end{proof}

\medskip

\begin{theorem}\label{th_13}
Let $A$ be a countable set, and let $E_n$ be the set of all $n$-tuples $(a_1, \cdots, a_n)$, where $a_k \in A, k = 1, \cdots, n$, and the elements $a_1, \cdots, a_n$ need not be distinct. Then $E_n$ is countable.
\end{theorem}
\begin{proof}
$E_1$ is countable since $E_1 = A$. Suppose $E_{n-1}$ is countable, Then the elements of $E_n$ has form $(b,a)$, where $b \in E_{n-1}, a \in A$. For every fixed $b$, the set of pairs $(b,a)$ is equivalent to $A$, hence countable. Thus, $E_n$ is a countable union of countable sets, hence countable by Theorem \ref{th_12}.
\end{proof}

\medskip

\begin{theorem}
The set of rational numbers $\mathbb{Q}$ and $\mathbb{N}$ have the same cardinality.
\end{theorem}
\begin{proof}
Each rational number $q \in \mathbb{Q}$ can be expressed as a quotient $n/m$ where $n \in \mathbb{Z}$ and $m \in \mathbb{N}$ and the greatest common divisor of $\left|n\right|$ and $m$ is $1$. Let $f: \mathbb{Q} \to \mathbb{Z}^2$ be defined by $f(q) = (n,m)$. Then the set of all pairs $(n,m)$ is countable by Theorem \ref{th_13}.
\end{proof}

\medskip

Not all infinite sets are countable. The example of uncountable set is shown below.

\medskip

\begin{theorem}
Let $A$ be the set of all sequences whose elements are the digits $0$ and $1$. The set $A$ is uncountable.
\end{theorem}
\begin{proof}
The elements of $A$ are of form $1,0,1,1,0,\cdots$. Let $E$ be a countable subset of $A$ and let $E = \{s_k\}$. We construct a sequence $s$ as follows, if $n$th digit in $s_n$ is $1$, then let $n$th digit of $s$ be $0$, vice versa. Then the sequence $s$ differs from every element of $E$, hence $s \notin E$. However, $s \in A$, then $E$ is a proper subset of $A$.

Now we have proved that every countable subset of $A$ is a proper subset of $A$, thus $A$ is uncountable. Otherwise, $A$ would be a proper subset of $A$, which is a contradiction. 
\end{proof}

\begin{remark}
This theorem implies that with the binary representation, the set of all real numbers is uncountable.
\end{remark}

\medskip


\section{Metric spaces}

\begin{definition}
A set $X$, whose elements are called points, is said to be a metric space if with any two points $x$ and $y$ of $X$ there is associated a real number $d(x,y): \mathbb{R}^n \times \mathbb{R}^n \rightarrow \mathbb{R}$ called the distance from $x$ to $y$, which is defined as 
\begin{align*}
    d(x,y) = \left|x - z\right|,
\end{align*}
which has the following properties
\begin{enumerate}[label=(\alph*)]
    \item $d(x,y) > 0$ if $x\neq y$,
    \item $d(x,y) = 0$ if $x = y$,
    \item $d(x,y) = d(y,x)$,
    \item $d(x,y) \leq d(x,z) + d(z,y)$.
\end{enumerate}
Any function with these three properties is called a distance function, or a metric. And the pair $(X,d)$ is called the metric space.
\end{definition}

\medskip

\begin{example}
Examples of metric spaces:
\begin{enumerate}[label=(\alph*)]
    \item $(\mathbb{R}^n,\rho_1)$, where $\rho_1(x,y) = \max_i \left|x_i - y_i\right|$.
    
    \item $(\mathbb{R}^n,\rho_2)$, where $\rho_2(x,y) = \sum^n_{i=1} |x_i-y_i|$, this is called taxi metric or New York metric. 
    
    \item $(\mathbb{R}^n,\rho_3)$, where $\rho_3(x,y) = \left\|x - y\right\| = \left(\sum^n_{i=1} (x_i - y_i)^2 \right)^{1/2}$, this is called standard Euclidean space.
    
    
    \item $(X,d)$, where $X$ is arbitrary set and 
    \begin{align*}
        d(x,y) = \begin{cases}
            1, & x \neq y, \\
            0, & x = y.
        \end{cases}
    \end{align*} 
    This is called discrete metric space.
    
    \item For $x = \{x_n\}^\infty_{n=1}$, let $l^1 = \{x \,:\, \sum^\infty_{n=1}\left|x_n\right| < \infty\}$, i.e., $l^1$ is the space of all absolutely convergent sequences. For $x, y \in l^1$, we define 
    \begin{align*}
        d_1(x,y) = \sum^\infty_{n=1} \left|x_n - y_n\right|.
    \end{align*}
    Then $(l^1,d_1)$ is a metric space.
    
    \item For $x = \{x_n\}^\infty_{n=1}$, let $l^2 = \{x  \,:\, \sum^\infty_{n=1}\left|x_n\right|^2 < \infty\}$. For $x, y \in l^1$, we define
    \begin{align*}
        d_2(x,y) = \left(\sum^\infty_{n=1} (x_n - y_n)^2\right)^{1/2}.
    \end{align*}
    Then $(l^2, d_2)$ is a metric space and this space is call Hilbert space.
\end{enumerate}
\end{example} 

\medskip

\begin{definition}
By the segment $(a,b)$ we mean the set if all real numbers $x$ such that $1 < x < b$. By the interval $[a,b]$ we mean the set of all real numbers $x$ such that $1 \leq x \leq b$. 

If $a_i < b_i$ for $i = 1, \cdots, k$, the set of all points $x = (x_1, \cdots, x_k) \in \mathbb{R}^k$ such that $a_i \leq x_i \leq b_i$ for $i = 1, \cdots, k$ is call a $k$-cell. 

We call a set $E \subset \mathbb{R}^k$ convex if $\lambda x + (1 - \lambda)y \in E$, for any $x,y \in E$ and $0 < \lambda < 1$.
\end{definition}

\medskip

\begin{definition}
Let $X$ be a metric space. 
\begin{enumerate}[label=(\alph*)]
    \item A neighborhood or a ball of $x$ is a set $B(x,r)$ consisting of all $y$ such that $d(x,y) < r$, for some $r > 0$. The number $r$ is called the radius of $B(x,r)$.
    
    \item A point $x$ is a limit point of set $E$ if every neighborhood of $x$ contains a point $y \neq x$ such that $y \in E$.
    
    \item If $x\in E$ and $x$ is not a limit point of $E$, then $x$ is called an isolated point of $E$.
    
    \item $E$ is closed if every limit point of $E$ is a point of $E$.
    
    \item A point $x$ is an interior point of $E$ if there is a neighborhood (or ball) $B(x,r)$ of $x$ such that $B(x,r) \in E$.
    
    \item $E$ is open if every point of $E$ is an interior point of $E$.
    
    \item The complement of $E$ (denoted by $E^c$) is the set of all points $x\in X$ such that $x\notin E$.
    
    \item $E$ is perfect if $E$ is closed and if every point of $E$ is a limit point of $E$.
    
    \item $E$ is bounded if there is a real number $M$ and a point $x\in X$ such that $d(x, y) < M$ for all $y\in E$.
    
    \item $E$ is dense in $X$ if every point of $X$ is a limit point of $E$, or a point of $E$(or both).
\end{enumerate}
\end{definition}

\medskip

\begin{theorem}
Every neighborhood is an open set.
\end{theorem}
\begin{proof}
Consider a neighborhood $B(x,r)$, it suffices to show that every point of $B$ is an interior point of $B$. Let $y$ be any point of $B(x,r)$, then there is $h > 0$ such that $d(x,y) = r - h$. For all points $s$ such that $d(y,s) < h$, we have
\begin{align*}
    d(x,s) \leq d(x,y) + d(y,s) \leq r - h + h = r,
\end{align*}
which implies $s \in E$. Thus $y$ is an interior point of $E$.
\end{proof}

\medskip

\begin{theorem}
If $x$ is a limit point of a set $E$, then every neighborhood of $x$ contains infinitely many points of $E$.
\end{theorem}
\begin{proof}
Suppose there is a neighborhood $B$ of $x$ which only contains only a finite number of points of $E$. Let $B \cap E = \{x_1, \cdots, x_n\}$, which are distinct from $x$, and let 
\begin{align*}
    r = \min_{1\leq i\leq n} d(x, x_n).
\end{align*}
Clearly, $r$ exists and $r > 0$. Hence, $B(x,r)$ contains no point of $E$ which is distinct from $x$, and then $x$ is not a limit point, a contradiction. 
\end{proof}

\medskip

\begin{corollary}
A finite point set has no limit point.
\end{corollary}

\medskip

\begin{theorem}\label{th_18}
Let $\{E_{\alpha}\}$ be a (finite or infinite) collection of sets $E_{\alpha}$. Then,
\begin{align*}
    \left(\bigcup_{\alpha} E_{\alpha}\right)^c = \bigcap_{\alpha} (E_{\alpha}^c).
\end{align*}
\end{theorem}
\begin{proof}
If $x \in \left(\bigcup_{\alpha} E_{\alpha}\right)^c$, then $x \notin \bigcup_{\alpha} E_{\alpha}$, hence $x \notin E_{\alpha}$ for every $\alpha$. Hence, $x \in E_{\alpha}^c$ for every $\alpha$, so $x \in \bigcap_{\alpha} (E_{\alpha}^c)$.
On the other hand, if $x \in \bigcap_{\alpha} (E_{\alpha}^c)$, then $x \in E_{\alpha}^c$ for every $\alpha$, hence $x \notin E_{\alpha}$ for every $\alpha$. Hence, $x \notin \bigcup_{\alpha} E_{\alpha}$, and thus, $x \in \left(\bigcup_{\alpha} E_{\alpha}\right)^c$. 
\end{proof}

\medskip

\begin{theorem}\label{th_19}
A set $A$ is open if and only if it complement is closed.
\end{theorem}
\begin{proof}
First, suppose $A^c$ is closed. For $x\in A$, then $x\notin A^c$, and $x$ is not a limit point of $E^c$. Then there exists $r>0$ such that $B(x,r) \cap A^c = \varnothing$. Then, we have $B(x,r) \subset A$. Thus $x$ is an interior point of $A$ and it follows that $A$ is open.

Next, suppose $A$ is open. Let $x$ be a limit point of $A^c$. Then every neighborhood of $x$ contains a point of $A^c$, so $x$ is not a interior point of $A$. Since $A$ is open, then $x\notin A$, which means $x\in A^c$. Since $x$ is a limit point of $A^c$, then $A^c$ is closed.
\end{proof}

\medskip

\begin{corollary}
A set $F$ is closed if and only if its complement is open.
\end{corollary}

\medskip

\begin{theorem}\label{th_110}
~\begin{enumerate}[label=(\alph*)]
    \item For any collection $\{G_{\alpha}\}$ of open sets, $\bigcup_{\alpha} G_{\alpha}$ is open.\label{th_110_a}
    
    \item For any collection $\{F_{\alpha}\}$ of closed sets, $\bigcap_{\alpha} F_{\alpha}$ is closed.\label{th_110_b}
    
    \item For any finite collection $G_1, \cdots, G_n$ of open sets, $\bigcap^n_{i=1} G_i$ is open.\label{th_110_c}
    
    \item For any finite collection $F_1, \cdots, F_n$ of closed sets, $\bigcup^n_{i=1} F_i$ is closed.\label{th_110_d}
\end{enumerate}
\end{theorem}
\begin{proof}
Let $G = \bigcup_{\alpha} G_{\alpha}$. If $x \in G$, then $x \in G_{\alpha}$ for some $\alpha$. Since $G_{\alpha}$ is open, then $x$ is an interior point of $G_{\alpha}$, and of course an interior point of $G$. Hence $G$ is open, and this proves \ref{th_110_a}.

By Theorem \ref{th_18}, 
\begin{align}\label{th_110_equ_1}
    \left(\bigcap_{\alpha} F_{\alpha}\right)^c = \bigcup_{\alpha} \left(F_{\alpha}^c\right),
\end{align}
and $F_{\alpha}^c$ is open by Theorem \ref{th_19}. Also, \ref{th_110_a} implies that \eqref{th_110_equ_1} is open and thus $\bigcap_{\alpha} F_{\alpha}$ is closed.

Now let $H = \bigcap^n_{i=1} G_i$, for any $x \in H$, there is neighborhoods $B(x,r_i)$ of $x$ such that each $B(x,r_i) \subset G_i, i = 1, \cdots, n$. Let $r = \min \{r_1, \cdots, r_n\}$, then clearly $B(x,r) \subset G_i$ for every $i = 1, \cdots, n$, and hence $B(x,r) \subset G$. Thus $G$ is open.

Similarly, \ref{th_110_d} follows from \ref{th_110_c} by
\begin{align*}
    \left(\bigcup^n_{i=1} F_{\alpha}\right)^c = \bigcap^n_{i=1} \left(F_{\alpha}^c\right)
\end{align*}
\end{proof}

\begin{remark}
In \ref{th_110_c} and \ref{th_110_d}, the finiteness of the union and intersection is required. For example, for $n = 1,2,3,\cdots$, let 
\begin{align*}
    G_n = \left(- \frac{1}{n}, \frac{1}{n}\right), 
\end{align*}
then $G = \bigcap^\infty_{n=1} G_n = \{0\}$, which is closed.  

Similarly, for $n = 1,2,3,\cdots$, let  
\begin{align*}
    F_n = \left[\frac{1}{n}, 1 - \frac{1}{n}\right],
\end{align*}
then $F = \bigcup^\infty_{n=1} F_n = (0,1)$, which is not closed.
\end{remark}

\medskip

\begin{definition}
Given $A \subset X$, the interior of the set $A$ is defined as the set of all points $x \in A$ that has a neighborhood contained in $A$, that is
\begin{align*}
    \operatorname{int}(A) = \{x \in A \,:\, \exists \, r > 0, B(x,r) \subset A\}.
\end{align*}
\end{definition}

\medskip

\begin{theorem}
The interior $\operatorname{int}(A)$ is always open and it is the largest open set contained in $A$ in the sense that if $G \subset A$ is open, then $G \subset \operatorname{int}(A)$.
\end{theorem}
\begin{proof}
Clearly, $\operatorname{int}(A)$ is open. Indeed, if $x \in \operatorname{int}(A)$, then there is $r > 0$ such that $B(x,r) \subset A$. Now consider any $y \in B(x,r/2)$, then $d(x,y) < r/2$. Then for any $z \in B(y, r/2)$, we have
\begin{align*}
    d(x,z) \leq d(x,y) + d(y,z) < r,
\end{align*}
which implies that every point in $B(x,r/2)$ is an interior point of $A$. Hence, there is a neighborhood $B(x,r/2)$ of $x$ such that $B(x,r/2) \subset \operatorname{int}(A)$, thus $\operatorname{int}(A)$ is open.

Now, let $G \subset A$ be any open set. Then for any $x \in G$, there is $r > 0$ such that $B(x,r) \subset G \subset A$, hence $x \in \operatorname{int}(A)$. Thus $G \subset \operatorname{int}(A)$.
\end{proof}

\medskip

\begin{definition}
If $X$ is a metric space, if $E \subset X$ and $E'$ denotes the set of all limit points of $E$, then the closure of $E$ is the set $\overline{E} = E \cup E'$.
\end{definition}

\medskip

\begin{theorem}
If $X$ is a metric space and $E \subset X$, then
\begin{enumerate}[label=(\alph*)]
    \item $\overline{E}$ is closed, \label{th_112_a}
    
    \item $E = \overline{E}$ if and only if $E$ is closed, \label{th_112_b}
    
    \item $\overline{E} \subset F$ for every closed set $F \subset X$ that contains $E$. \label{th_112_c}
\end{enumerate}
By \ref{th_112_a} and \ref{th_112_c}, $\overline{E}$ is the smallest closed subset of $X$ that contains $E$.
\end{theorem}
\begin{proof}
~\begin{enumerate}[label=(\alph*)]
    \item If $x \notin \overline{E}$, then $x$ is neither a point of $E$ nor a limit point of $E$. Then there is a neighborhood $B$ of $x$ such that $B \cap E = \emptyset$. Hence, $\overline{E}^c$ is open, and thus $\overline{E}$ is closed. 
    
    \item If $E = \overline{E}$, then \ref{th_112_a} implies $E$ is closed. If $E$ is closed, then $E' \subset E$, hence $\overline{E} = E \cup E' = E$.
    
    \item If $F$ is closed and $E \subset F$, then $F' \subset F$, hence $E' \subset F'$. Thus $\overline{E} \subset F$.
\end{enumerate}
\end{proof}

\medskip

\begin{theorem}
Let $E$ be a nonempty set of real numbers which is bounded above. Let $y = \sup E$.\footnote{Recall the {\em least upper bound} or the {\em supremum}. Suppose $S$ is an ordered set, $E \subset S$ and $E$ is bounded above. Then the supremum $\alpha \in S$ satisfies the following properties: \begin{enumerate*}
    \item[(i)] $\alpha$ is an upper bound of $E$,
    \item[(ii)] If $\gamma < \alpha$, then $\gamma$ is not an upper bound of $E$.
\end{enumerate*} And we write $\alpha = \sup E$. The {\em greatest lower bound} or the {\em infimum} is defined in a similar way.}Then $y \in \overline{E}$. Hence $y \in E$ if $E$ is closed.
\end{theorem}
\begin{proof}
If $y \in E$, then $y \in \overline{E}$. Assume that $y \notin E$, for every $\varepsilon > 0$, there exists $x \in E$ such that $y - \varepsilon < x < y$, otherwise $y - \varepsilon$ is an upper bound of $E$. Hence, $y$ is a limit point of $E$, and thus $y \in \overline{E}$.
\end{proof}

\begin{remark}
Suppose $E \subset Y \subset X$. We say that $E$ is {\em open relative} to $Y$ if to each $x \in E$, there is $r > 0$ such that $y \in E$ whenever $d(x,y) < r$ and $y \in Y$. We talk about this since a set may be open relative to $Y$ without being an open subset of $X$. For example, let $E = (a,b)$, $a < b$ and $a,b \in \mathbb{R}$, $Y = \mathbb{R}$ and $X = \mathbb{R}^2$, then $(a,b)$ is an open subset of $\mathbb{R}$, but not an open subset of $\mathbb{R}^2$.
\end{remark}

\medskip

\begin{theorem}
Suppose $Y \subset X$. A subset $E$ of $Y$ is open relative to $Y$ if and only if $E = Y \cap G$ for some open subset $G$ of $X$.
\end{theorem}
\begin{proof}
Suppose $E$ is open relative to $Y$. For each $x \in E$, there is $r_x > 0$ such that $y \in E$ if $d(x,y) < r_x$ and $y \in Y$. Let $G_x = \{y \in X \,:\, d(x,y) < r_x\}$, and define
\begin{align*}
    G = \bigcup_{x \in E} G_x.
\end{align*}
Then $G$ is an open subset of $X$ by Theorem \ref{th_110}. Clearly, $E \subset G \cap Y$. Also, $G_x \cap Y \subset E$ for every $x \in E$, so that $G \cap Y \subset E$. Hence, $E = G \cap Y$. 

Conversely, if $G$ is open in $X$ and $E = G \cap Y$, every $x \in E$ has a neighborhood $B_x \subset G$. Then, $B_x \cap Y \subset E$, and thus $E$ is open relative to $Y$.
\end{proof}

\medskip


\section{Compact Sets}

\begin{definition}
An open cover of a set $E$ in a metric space $X$ is a collection $\{G_{\alpha}\}$ of open subsets of $X$ such that $E \subset \bigcup_{\alpha} G_{\alpha}$.
\end{definition}

\medskip

\begin{definition}
A subset $K$ of a metric space $X$ is said to be compact if every open cover of $K$ contains a finite subcover. More explicitly, this requirement is that if $\{G_{\alpha}\}$ is an open cover of $K$, then there are finitely many $\alpha_1, \cdots, \alpha_n$ such that 
\begin{align*}
    K \subset \bigcup^n_{i=1} G_{\alpha_i}.
\end{align*}
\end{definition}

\begin{remark}
With the familiarity of continuity, compactness can also be defined as follows: $K \subset X$ is compact if every sequence in $K$ has subsequence converging to a point in $K$. We will talk more about this later.
\end{remark}


















\newpage
\bibliographystyle{unsrt}
\bibliography{bibliography}

\end{document}