\documentclass[11pt]{book}
\pagestyle{plain}
%\documentclass{article}

\usepackage[utf8]{inputenc}
\usepackage[english]{babel}
\usepackage[T1]{fontenc}
\usepackage{latexsym,amsmath,amssymb}
\usepackage{amsthm}
\usepackage{amsfonts}
\usepackage{geometry}
\usepackage{graphicx}
\usepackage{lmodern}
\usepackage{pifont}
\usepackage{tikz}
\usepackage{pgfplots}
\usepackage{thmtools}
\usepackage{wrapfig}
\usepackage{extarrows}
\usepackage{breqn}
\usepackage{physics}
\usepackage{afterpage}
\usepackage[inline]{enumitem}
\usepackage{mathrsfs}
\usepackage{scalerel}
\usepackage{stackengine,wasysym}
\usepackage{aligned-overset}
\usepackage{stackengine}
\usepackage{mathtools}
\usepackage{nccmath}
\usepackage{url}
\usepackage{float}
\usepackage{lipsum}
\usepackage[toc]{appendix}
\usepackage{chngcntr}
\usepackage{etoolbox}
\usepackage{framed}
\usepackage{mdframed}
\usepackage{blindtext}
\usepackage{xcolor}
\usepackage{fancyhdr}
\usepackage{titlesec}
\usepackage{esint}
\usepackage{aligned-overset}
\usepackage{mathrsfs}
\graphicspath{{images/}}

\DeclarePairedDelimiter\ceil{\lceil}{\rceil}
\DeclarePairedDelimiter\floor{\lfloor}{\rfloor}


\titleformat{\chapter}[block]{\huge\bfseries\itshape\raggedleft}{\chaptertitlename\  \thechapter.}{0.5ex}{}[]

\titleformat{\section}[block]{\Large\bfseries\itshape}{\thesection\ }{0.5ex}{}[]

\titleformat{\subsection}[block]{\large\bfseries\itshape}{\thesubsection\ }{0.5ex}{}[]

\usepackage{hyperref}
\hypersetup{
    colorlinks = true,
    linkcolor = blue,
    filecolor = blue,      
    urlcolor = blue,
    citecolor = blue,
    pdftitle = {Sharelatex Example},
    bookmarks = true,
    pdfpagemode = FullScreen,
}

\urlstyle{same}

\newcommand*\circled[1]{\tikz[baseline=(char.base)]{
            \node[shape=circle,draw,inner sep=2pt] (char) {#1};}}

\setlength{\oddsidemargin}{1pt}
\setlength{\evensidemargin}{1pt}
\setlength{\marginparwidth}{30pt} % these gain 53pt width
\setlength{\topmargin}{1pt}       % gains 26pt height
\setlength{\headheight}{1pt}      % gains 11pt height
\setlength{\headsep}{1pt}         % gains 24pt height
%\setlength{\footheight}{12 pt} 	  % cannot be changed as number must fit
\setlength{\footskip}{24pt}       % gains 6pt height
\setlength{\textheight}{650pt}    % 528 + 26 + 11 + 24 + 6 + 55 for luck
\setlength{\textwidth}{460pt}     % 360 + 53 + 47 for luck

\title{Sections and Chapters}

\newtheorem{definition}{Definition}[chapter]
\newtheorem{theorem}{Theorem}[chapter]
\newtheorem{corollary}{Corollary}[theorem]
\newtheorem{lemma}{Lemma}[chapter]
\newtheorem{proposition}{Proposition}[chapter]
\newtheorem{exercise}{Exercise}[section]
\newtheorem{remark}{Remark}[chapter]
\theoremstyle{definition}
\newtheorem{example}{Example}[chapter]
\numberwithin{equation}{chapter}


\AtBeginEnvironment{subappendices}{%
\Appendix*{Appendix}
\addcontentsline{toc}{chapter}{Appendices}
\counterwithin{figure}{section}
\counterwithin{table}{section}
}

\newmdenv[
  linewidth=1.5pt, 
  topline=false, 
  bottomline=false, 
  rightline=false,
  innerleftmargin=15pt,
  leftmargin=10pt,
  rightmargin=0pt,
  innerrightmargin=0pt, 
]{claim}

\def\MM{\mathfrak{M}}
\def\BB{\mathfrak{B}}
\def\CC{\mathfrak{C}}
\def\leb{{\mathcal L}}
\def\H{{\mathcal H}}
\def\L{{\mathcal L}}
\def\diam{{\operatorname{diam}\,}}
\def\co{{\overline{\operatorname{co}}\,}}
\def\dist{{\operatorname{dist}\,}}

\usepackage{scalerel}
%\usepackage[usestackEOL]{stackengine}
\def\avint{\,\ThisStyle{\ensurestackMath{%
  \stackinset{c}{.2\LMpt}{c}{1\LMpt}{\SavedStyle-}{\SavedStyle\phantom{\int}}}%
  \setbox0=\hbox{$\SavedStyle\int\,$}\kern-\wd0}\int}

\pagestyle{fancy}
\fancyhf{}
\fancyhead[LE]{{\em \thepage}}
\fancyhead[RE]{{\em \leftmark}}
\fancyhead[LO]{{\em \rightmark}}
\fancyhead[RO]{{\em \thepage}}


\counterwithout{footnote}{chapter}

\def\dsp{\def\baselinestretch{1.35}\large
\normalsize}
%%%%This makes a double spacing. Use this with 11pt style. If you
%%%%want to use this just insert \dsp after the \begin{document}
%%%%The correct baselinestretch for double spacing is 1.37. However
%%%%you can use different parameter.

\newcommand\blankpage{%
    \null
    \thispagestyle{empty}%
    \addtocounter{page}{-1}%
    \newpage}
    
\def\U{{\mathcal U}}

\begin{document}
\frontmatter

\begin{titlepage}
	\begin{center}
	\textbf{\LARGE{}} \\
	\vspace{40mm}
    \textbf{\Huge{Mathematical Analysis}} \\
    \medskip
    \vspace{10mm} %5mm vertical space
    \large{\textsc{Zhen Yao}}\\
    %\large{\textsc{University of Pittsburgh}}
    \end{center}
\end{titlepage}

\tableofcontents{}
\mainmatter

\newpage

\chapter*{Preface}
\addcontentsline{toc}{chapter}{Preface}

The main purpose of this book is to help prepare for the preliminary exam in Mathematical Analysis as a new mathematics PhD student. This book will follow the structure in Rudin's {\it Principles of Mathematical Analysis}\cite{1} and more contents will be added in each chapters, some are from Dr. Piotr Hajłasz's {\it Introduction to Analysis}\cite{2}.

The content of real and complex number system, especially how to constructing the real number system, will not be covered in this book, since it is perfectly discussed in both Rudin's and Dr. Hajłasz's books. Thus, this book starts with the basic topology. 

\null\hfill ZHEN YAO

\newpage

\chapter{Basic Topology}

\section{Functions}

\begin{definition}
Let $X$ and $Y$ be two sets, and suppose with each element $x$ of $X$ there is associated an element of $Y$, denoted by $f(x)$. Then $f$ is said to to a function from $X$ to $Y$ (or a mapping of $X$ into $Y$), and we write it as $f: X \to Y$. Also,
\begin{enumerate}[label=(\alph*)]
    \item the set $X$ is called the domain of $f$, and $f$ is said to be defined on $X$;
    
    \item the set $Y$ is called the target of $f$;
    
    \item the set of all values of $f(X) \coloneqq \{y \in Y \,:\, \exists x \in X, y = f(x)\}$ is called the range of $f$. If $E \subset X$, we call $f(E)$ the image of $E$ under $f$, and clearly $f(E) \subset Y$;
    
    \item the graph of $f$ is a subset of $X \times Y$, defined by
    \begin{align*}
        \operatorname{graph}(f) \coloneqq \left\{(x,f(x)) \in X \times Y \,:\, x \in X \right\}.
    \end{align*}
\end{enumerate}
\end{definition}

\medskip

\begin{definition}
Let $f: X \to Y$. If $f(X) = Y$, we say $f$ is onto or surjective if $f(X) = Y$. Clearly, the term of onto is more specific than into.

If $E \subset Y$, $f^{-1}(E)$ denotes the set of all $x \in X$ such that $f(x) \in E$. We call $f^{-1}(E)$ the inverse image of $E$ under $f$. If for each $y \in Y$, $f^{-1}(y)$ consists of at most one element of $X$, then $f$ is said to be one-to-one or injective, i.e. if for any $x_1 \neq x_2, x_1, x_2 \in X$, then $f(x_1) \neq f(x_2)$.
\end{definition}

\medskip

\begin{definition}
A function $f:X \to Y$ that is one-to-one and onto is called a bijection. In this case, the inverse function is defined on $Y = f(X)$ and the inverse function $f^{-1}: Y \to X$ is also a bijection.
\end{definition}

\begin{definition}
If $f: X \to Y$ and $g: Y \to Z$ are two functions, then the composition of $f$ and $g$ is the function $g \circ f: X \to Z$ defined by $(g \circ f)(x) = g(f(x))$ for $x \in X$.
\end{definition}

\medskip

\begin{definition}
If $f: X \to Y$ and $A \subset X$, then the restriction of $f$ to $A$ is $f|_A: A \to Y$ defined by $f|_A(x) = f(x)$ for all $x \in A$. In this case, we say that $f$ is an extension of $f|_A$.
\end{definition}

\medskip

\begin{proposition}
Let $f: X \to Y$ be a function and let $A, B \subset Y$. Then,
\begin{align*}
    f^{-1}(A \cup B) & = f^{-1}(A) \cup f^{-1}(B), \\
    f^{-1}(A \cap B) & = f^{-1}(A) \cap f^{-1}(B), \\
    f^{-1}(Y \setminus A) & = X \setminus f^{-1}(A).
\end{align*}
\end{proposition}

\begin{remark}
Given a set $E \subset X$, $E^c = X \setminus E$ denotes the complement of the set $E$ in $X$. With this notation, the last equality can be expressed as
\begin{align*}
    f^{-1}(A^c) = \left(f^{-1}(A)\right)^c.
\end{align*}
\end{remark}
\begin{proof}
First, if $x \in f^{-1}(A \cup B)$, then $f(x) \in A \cup B$, and hence $f(x) \in A$ or $f(x) \in B$. If $f(x) \in A$, then clearly $x \in f^{-1}(A) \subset f^{-1}(A) \cup f^{-1}(B)$. If $f(x) \in B$, then clearly $x \in f^{-1}(B) \subset f^{-1}(A) \cup f^{-1}(B)$. And hence $f^{-1}(A \cup B) \subset f^{-1}(A) \cup f^{-1}(B)$. On the other hand, if $x \in f^{-1}(A) \cup f^{-1}(B)$, then $x \in f^{-1}(A)$ or $x \in f^{-1}(B)$. If $x \in f^{-1}(A)$, then $f(x) \in A \subset A \cup B$. If $x \in f^{-1}(B)$, then $f(x) \in B \subset A \cup B$. In either case $f(x) \in A \cup B$, and hence $x \in f^{-1}(A \cup B)$. This completes the proof of the first equality.

Second, if $x \in f^{-1}(A \cap B)$, then $f(x) \in A \cap B$, hence $f(x) \in A$ and $f(X) \in B$. In this case, $x \in f^{-1}(A) \cap f^{-1}(B)$. On the other hand, if $x \in f^{-1}(A) \cap f^{-1}(B)$, then $f(x) \in A$ and $f(x) \in B$. Hence, $f(x) \in A \cap B$ and this implies $x \in f^{-1}(A \cap B)$. This completes the proof of the second equality.

Finally, if $x \in f^{-1}(Y \setminus A)$, then $f(x) \in Y \setminus A$, and hence $x \notin f^{-1}(A)$. This implies $x \in X \setminus f^{-1}(A)$. On the other hand, if $x \in X \setminus f^{-1}(A)$, then $f(x) \notin A$, and hence $f(x) \in Y \setminus Y$, and this implies $x \in f^{-1}(Y \setminus A)$. 
\end{proof}

\medskip

\begin{proposition}
Let $f: X \to Y$ be a function and let $A, B \subset X$. Then,
\begin{align*}
    f(A \cup B) = f(A) \cup f(B), \qquad f(A \cap B) \subset f(A) \cap f(B).
\end{align*}
If in addition $f$ is one-to-one then
\begin{align*}
    f(A \cap B) = f(A) \cap f(B).
\end{align*}
\end{proposition}
\begin{proof}
First, if $y \in f(A \cup B)$, then $y = f(x)$ for some $x \in A \cup B$. If $x \in A$, then $y = f(x) \in f(A)$. If $x \in B$, then $y = f(x) \in f(B)$. In either case, we have $y \in f(A) \cup f(B)$. On the other hand, if $y \in f(A) \cup f(B)$, then $y \in f(A)$ or $y \in f(B)$. If $y \in f(A)$, then $y = f(x)$ for some $x \in A$. If $y \in f(B)$, then $y = f(x)$ for some $x \in B$. In either case, we have $y = f(x) \in f(A \cup B)$.

Second, if $y \in f(A \cap B)$, then $y = f(x)$ for some $x \in A \cap B$. Then, $y = f(x) \in f(A)$ and $y = f(x) \in f(B)$, hence $y \in f(A) \cap f(B)$. This implies $f(A \cap B) \subset f(A) \cap f(B)$.

Finally, if $f$ is one-to-one, it remains to show that $f(A) \cap f(B) \subset f(A \cap B)$. If $y \in f(A) \cap f(B)$, then there exist $x_1 \in A$ and $x_2 \in B$ such that $y = f(x_1)$ and $y = f(x_2)$. Since $f$ is one-to-one, then $x_1 = x_2$ and clearly $x_1 = x_2 \in A \cap B$, and hence $y = f(x_1) = f(x_2) \in f(A \cap B)$. This completes the proof of $f(A) \cap f(B) \subset f(A \cap B)$.
\end{proof}

\begin{remark}
Why $f(A) \cap f(B) \subset f(A \cap B)$ fail to hold when $f$ is not one-to-one? If $y \in f(A) \cap f(B)$, then $y = f(x_1)$ for some $x_1 \in A$ and $y = f(x_2)$ for some $x_2 \in B$. However, $x_1$ may not be equal to $x_2$ and hence we cannot claim there is a common $x \in A \cap B$ such that $y = f(x)$. And this proposition leads to the following results.
\end{remark}

\medskip

\begin{proposition}
Let $f: X \to Y$ be a function and $A_1, A_2, A_3, \cdots$ are subsets of $X$, then
\begin{align*}
    f \left(\bigcup^\infty_{i=1} A_i\right) = \bigcup^\infty_{i=1} f(A_i), \qquad f \left(\bigcap^\infty_{i=1} A_i\right) \subset \bigcup^\infty_{i=1} f(A_i).
\end{align*}
If in addition $f$ is one-to-one, then
\begin{align*}
    f \left(\bigcap^\infty_{i=1} A_i\right) = \bigcup^\infty_{i=1} f(A_i).
\end{align*}
\end{proposition}


\medskip

\section{Finite, Countable, and Uncountable Sets}

\begin{definition}
If there is a one-to-one mapping of $X$ onto $Y$, we say that $X$ and $Y$ is one-to-one correspondence, or that $X$ and $Y$ have the same cardinal number, or $X$ and $Y$ are equivalent, and we write $X \sim Y$. Clearly, this relation has the following properties:
\begin{enumerate}[label=(\alph*)]
    \item It is reflexive: $X \sim X$;
    
    \item It is symmetric: If $X \sim Y$, then $Y \sim X$;
    
    \item It is transitive: If $X \sim Y$ and $Y \sim Z$, then $X \sim Z$.
\end{enumerate}
Any relation with these three properties is called an equivalence relation.
\end{definition}

\medskip

\begin{remark}
A bijection $f:X \to Y$ is a one-to-one correspondence between $X$ and $Y$.
\end{remark}

\medskip

\begin{definition}
For any positive integer $n \in \mathbb{N}$, where $\mathbb{N}$ is the set of all positive integers, let $J_n = \{1,2,\cdots,n\}$. For any set $A$, we say:
\begin{enumerate}[label=(\alph*)]
    \item $A$ is finite if $A \sim J_n$ for some $n$ (the empty set $\emptyset$ is also considered to be finite).
    
    \item $A$ is infinite if $A$ is not finite.
    
    \item $A$ is countable if $A \sim \mathbb{N}$ (Countable sets are sometimes called enumerable or denumerable).
    
    \item $A$ is at most countable if $A$ is finite or countable.
\end{enumerate}
\end{definition}

\medskip

\begin{proposition}
The sets $\mathbb{N}$ and $2\mathbb{N} \coloneqq \{2n \,:\, x \in \mathbb{N}\}$ have the same cardinality.
\end{proposition}
\begin{proof}
Indeed, let $f(x) = 2n$ for $n \in \mathbb{N}$, then $f: \mathbb{N} \to 2\mathbb{N}$ is clearly a bijection.
\end{proof}

\medskip

\begin{proposition}
The sets $\mathbb{N}$ and $\mathbb{Z}$ have the same cardinality.
\end{proposition}
\begin{proof}
Indeed, for $n \in \mathbb{N}$, let
\begin{align*}
    f(n) = \begin{cases}
        \frac{n}{2}, & n \,\,\text{even}, \\
        - \frac{n-1}{2}, & n \,\,\text{odd}.
    \end{cases}
\end{align*}
Clearly, $f: \mathbb{N} \to \mathbb{Z}$ is clearly a bijection.
\end{proof}

\medskip

\begin{definition}
A sequence is a function $f$ defined on the set $\mathbb{N}$. If $f(n) = x_n$, for $n \in \mathbb{N}$, denote the sequence $f$ by ${x_n}$, or sometimes by $\{x_1, x_2, x_3, \cdots\}$. The values of $f$, that is, $x_n$, are called the terms of the sequence. If $A$ is a set and if $x_n \in A$ for all $n \in \mathbb{N}$, then $\{x_n\}$ is said to be a sequence in $A$, or a sequence of elements of $A$.

Since every countable set is the range of a one-to-one function defined on $\mathbb{N}$, we may regard every countable set as the range of a sequence of distinct terms.
\end{definition}

\medskip

\begin{theorem}\label{th_11}
Every infinite subset of a countable set $A$ is countable.
\end{theorem}
\begin{proof}
Suppose $E \subset A$, and $E$ is infinite. Arrange the elements of $A$ in a sequence $\{x_n\}$. Now, let $n_{1}$ be the smallest integer such that $x_{n_1} \in E$. After choosing $n_1, \cdots, n_{k-1}$, let $n_k$ be the smallest integer greater than $n_{k-1}$ such that $x_{n_k} \in E$. Let $f(k) = x_{n_k}$ for $k \in \mathbb{N}$, we have a one-to-one correspondence between $E$ and $\mathbb{N}$.
\end{proof}

\medskip

\begin{theorem}\label{th_12}
Let $\{E_n\}, n \in \mathbb{N}$ be a sequence of countable sequences and let 
\begin{align}\label{th_11_equ_1}
    S = \bigcup^\infty_{n=1} E_n.
\end{align}
Then $S$ is countable.
\end{theorem}
\begin{proof}
Let every set $E_n$ be arranged in a sequence $\{x_{nk}\}, k \in \mathbb{N}$, and consider the infinite array:
\begin{align*}
    x_{11}, \quad x_{12}, \quad x_{13}, \quad x_{14}, \quad \cdots \\
    x_{21}, \quad x_{22}, \quad x_{23}, \quad x_{24}, \quad \cdots \\
    x_{31}, \quad x_{32}, \quad x_{33}, \quad x_{34}, \quad \cdots
\end{align*}
where the elements of $E_n$ form the $n$th row. These elements can be arranged into a sequence
\begin{align*}
    x_{11}, x_{21}, x_{12}, x_{31}, x_{22}, x_{13}, x_{41}, x_{32}, x_{23}, x_{14}, \cdots
\end{align*}
If any two of the set $E_n$ have elements in common, then there is a subset $T$ of $\mathbb{N}$ such that $S \sim T$, hence by Theorem \ref{th_11}, $S$ is at most countable. Also, $E_1 \subset S$ is infinite, then $S$ is also infinite, and thus countable.
\end{proof}

\medskip

\begin{corollary}
Suppose $A$ is at most countable, and for every $\alpha \in A$, $E_{\alpha}$ is at most countable. Then,
\begin{align*}
    T = \bigcup_{\alpha \in A} E_{\alpha}
\end{align*}
is at most countable.
\end{corollary}
\begin{proof}
For $T$ is equivalent to a subset of \eqref{th_11_equ_1}.
\end{proof}

\medskip

\begin{theorem}\label{th_13}
Let $A$ be a countable set, and let $E_n$ be the set of all $n$-tuples $(a_1, \cdots, a_n)$, where $a_k \in A, k = 1, \cdots, n$, and the elements $a_1, \cdots, a_n$ need not be distinct. Then $E_n$ is countable.
\end{theorem}
\begin{proof}
$E_1$ is countable since $E_1 = A$. Suppose $E_{n-1}$ is countable, Then the elements of $E_n$ has form $(b,a)$, where $b \in E_{n-1}, a \in A$. For every fixed $b$, the set of pairs $(b,a)$ is equivalent to $A$, hence countable. Thus, $E_n$ is a countable union of countable sets, hence countable by Theorem \ref{th_12}.
\end{proof}

\medskip

\begin{theorem}
The set of rational numbers $\mathbb{Q}$ and $\mathbb{N}$ have the same cardinality.
\end{theorem}
\begin{proof}
Each rational number $q \in \mathbb{Q}$ can be expressed as a quotient $n/m$ where $n \in \mathbb{Z}$ and $m \in \mathbb{N}$ and the greatest common divisor of $\left|n\right|$ and $m$ is $1$. Let $f: \mathbb{Q} \to \mathbb{Z}^2$ be defined by $f(q) = (n,m)$. Then the set of all pairs $(n,m)$ is countable by Theorem \ref{th_13}.
\end{proof}

\medskip

Not all infinite sets are countable. The example of uncountable set is shown below.

\medskip

\begin{theorem}
Let $A$ be the set of all sequences whose elements are the digits $0$ and $1$. The set $A$ is uncountable.
\end{theorem}
\begin{proof}
The elements of $A$ are of form $1,0,1,1,0,\cdots$. Let $E$ be a countable subset of $A$ and let $E = \{s_k\}$. We construct a sequence $s$ as follows, if $n$th digit in $s_n$ is $1$, then let $n$th digit of $s$ be $0$, vice versa. Then the sequence $s$ differs from every element of $E$, hence $s \notin E$. However, $s \in A$, then $E$ is a proper subset of $A$.

Now we have proved that every countable subset of $A$ is a proper subset of $A$, thus $A$ is uncountable. Otherwise, $A$ would be a proper subset of $A$, which is a contradiction. 
\end{proof}

\begin{remark}
This theorem implies that with the binary representation, the set of all real numbers is uncountable.
\end{remark}

\medskip


\section{Metric spaces}

\begin{definition}
A set $X$, whose elements are called points, is said to be a metric space if with any two points $x$ and $y$ of $X$ there is associated a real number $d(x,y): \mathbb{R}^n \times \mathbb{R}^n \rightarrow \mathbb{R}$ called the distance from $x$ to $y$, which is defined as 
\begin{align*}
    d(x,y) = \left|x - z\right|,
\end{align*}
which has the following properties:
\begin{enumerate}[label=(\alph*)]
    \item $d(x,y) > 0$ if $x\neq y$;
    \item $d(x,y) = 0$ if $x = y$;
    \item $d(x,y) = d(y,x)$;
    \item $d(x,y) \leq d(x,z) + d(z,y)$.
\end{enumerate}
Any function with these three properties is called a distance function, or a metric. And the pair $(X,d)$ is called the metric space.
\end{definition}

\medskip

\begin{example}
Examples of metric spaces:
\begin{enumerate}[label=(\alph*)]
    \item $(\mathbb{R}^n,\rho_1)$, where $\rho_1(x,y) = \max_i \left|x_i - y_i\right|$.
    
    \item $(\mathbb{R}^n,\rho_2)$, where $\rho_2(x,y) = \sum^n_{i=1} |x_i-y_i|$, this is called taxi metric or New York metric. 
    
    \item $(\mathbb{R}^n,\rho_3)$, where $\rho_3(x,y) = \left\|x - y\right\| = \left(\sum^n_{i=1} (x_i - y_i)^2 \right)^{1/2}$, this is called standard Euclidean space.
    
    
    \item $(X,d)$, where $X$ is arbitrary set and 
    \begin{align*}
        d(x,y) = \begin{cases}
            1, & x \neq y, \\
            0, & x = y.
        \end{cases}
    \end{align*} 
    This is called discrete metric space.
    
    \item For $x = \{x_n\}^\infty_{n=1}$, let $l^1 = \{x \,:\, \sum^\infty_{n=1}\left|x_n\right| < \infty\}$, i.e., $l^1$ is the space of all absolutely convergent sequences. For $x, y \in l^1$, we define 
    \begin{align*}
        d_1(x,y) = \sum^\infty_{n=1} \left|x_n - y_n\right|.
    \end{align*}
    Then $(l^1,d_1)$ is a metric space.
    
    \item For $x = \{x_n\}^\infty_{n=1}$, let $l^2 = \{x  \,:\, \sum^\infty_{n=1}\left|x_n\right|^2 < \infty\}$. For $x, y \in l^1$, we define
    \begin{align*}
        d_2(x,y) = \left(\sum^\infty_{n=1} (x_n - y_n)^2\right)^{1/2}.
    \end{align*}
    Then $(l^2, d_2)$ is a metric space and this space is call Hilbert space.
\end{enumerate}
\end{example} 

\medskip

\begin{definition}
By the segment $(a,b)$ we mean the set if all real numbers $x$ such that $1 < x < b$. By the interval $[a,b]$ we mean the set of all real numbers $x$ such that $1 \leq x \leq b$. 

If $a_i < b_i$ for $i = 1, \cdots, k$, the set of all points $x = (x_1, \cdots, x_k) \in \mathbb{R}^k$ such that $a_i \leq x_i \leq b_i$ for $i = 1, \cdots, k$ is call a $k$-cell. 

We call a set $E \subset \mathbb{R}^k$ convex if $\lambda x + (1 - \lambda)y \in E$, for any $x,y \in E$ and $0 < \lambda < 1$.
\end{definition}

\medskip

\begin{definition}
Let $X$ be a metric space. 
\begin{enumerate}[label=(\alph*)]
    \item A neighborhood or a ball of $x$ is a set $B(x,r)$ consisting of all $y$ such that $d(x,y) < r$, for some $r > 0$. The number $r$ is called the radius of $B(x,r)$.
    
    \item A point $x$ is a limit point of set $E$ if every neighborhood of $x$ contains a point $y \neq x$ such that $y \in E$.
    
    \item If $x\in E$ and $x$ is not a limit point of $E$, then $x$ is called an isolated point of $E$.
    
    \item $E$ is closed if every limit point of $E$ is a point of $E$.
    
    \item A point $x$ is an interior point of $E$ if there is a neighborhood (or ball) $B(x,r)$ of $x$ such that $B(x,r) \in E$.
    
    \item $E$ is open if every point of $E$ is an interior point of $E$.
    
    \item The complement of $E$ (denoted by $E^c$) is the set of all points $x\in X$ such that $x\notin E$.
    
    \item $E$ is perfect if $E$ is closed and if every point of $E$ is a limit point of $E$.
    
    \item $E$ is bounded if there is a real number $M$ and a point $x\in X$ such that $d(x, y) < M$ for all $y\in E$.
    
    \item $E$ is dense in $X$ if every point of $X$ is a limit point of $E$, or a point of $E$(or both).
\end{enumerate}
\end{definition}

\medskip

\begin{theorem}
Every neighborhood is an open set.
\end{theorem}
\begin{proof}
Consider a neighborhood $B(x,r)$, it suffices to show that every point of $B$ is an interior point of $B$. Let $y$ be any point of $B(x,r)$, then there is $h > 0$ such that $d(x,y) = r - h$. For all points $s$ such that $d(y,s) < h$, we have
\begin{align*}
    d(x,s) \leq d(x,y) + d(y,s) \leq r - h + h = r,
\end{align*}
which implies $s \in E$. Thus $y$ is an interior point of $E$.
\end{proof}

\medskip

\begin{theorem}
If $x$ is a limit point of a set $E$, then every neighborhood of $x$ contains infinitely many points of $E$.
\end{theorem}
\begin{proof}
Suppose there is a neighborhood $B$ of $x$ which only contains only a finite number of points of $E$. Let $B \cap E = \{x_1, \cdots, x_n\}$, which are distinct from $x$, and let 
\begin{align*}
    r = \min_{1\leq i\leq n} d(x, x_n).
\end{align*}
Clearly, $r$ exists and $r > 0$. Hence, $B(x,r)$ contains no point of $E$ which is distinct from $x$, and then $x$ is not a limit point, a contradiction. 
\end{proof}

\medskip

\begin{corollary}
A finite point set has no limit point.
\end{corollary}

\medskip

\begin{theorem}\label{th_18}
Let $\{E_{\alpha}\}$ be a (finite or infinite) collection of sets $E_{\alpha}$. Then,
\begin{align*}
    \left(\bigcup_{\alpha} E_{\alpha}\right)^c = \bigcap_{\alpha} (E_{\alpha}^c).
\end{align*}
\end{theorem}
\begin{proof}
If $x \in \left(\bigcup_{\alpha} E_{\alpha}\right)^c$, then $x \notin \bigcup_{\alpha} E_{\alpha}$, hence $x \notin E_{\alpha}$ for every $\alpha$. Hence, $x \in E_{\alpha}^c$ for every $\alpha$, so $x \in \bigcap_{\alpha} (E_{\alpha}^c)$.
On the other hand, if $x \in \bigcap_{\alpha} (E_{\alpha}^c)$, then $x \in E_{\alpha}^c$ for every $\alpha$, hence $x \notin E_{\alpha}$ for every $\alpha$. Hence, $x \notin \bigcup_{\alpha} E_{\alpha}$, and thus, $x \in \left(\bigcup_{\alpha} E_{\alpha}\right)^c$. 
\end{proof}

\medskip

\begin{theorem}\label{th_19}
A set $A$ is open if and only if it complement is closed.
\end{theorem}
\begin{proof}
First, suppose $A^c$ is closed. For $x\in A$, then $x\notin A^c$, and $x$ is not a limit point of $E^c$. Then there exists $r>0$ such that $B(x,r) \cap A^c = \varnothing$. Then, we have $B(x,r) \subset A$. Thus $x$ is an interior point of $A$ and it follows that $A$ is open.

Next, suppose $A$ is open. Let $x$ be a limit point of $A^c$. Then every neighborhood of $x$ contains a point of $A^c$, so $x$ is not a interior point of $A$. Since $A$ is open, then $x\notin A$, which means $x\in A^c$. Since $x$ is a limit point of $A^c$, then $A^c$ is closed.
\end{proof}

\medskip

\begin{corollary}
A set $F$ is closed if and only if its complement is open.
\end{corollary}

\medskip

\begin{theorem}\label{th_110}
~\begin{enumerate}[label=(\alph*)]
    \item For any collection $\{G_{\alpha}\}$ of open sets, $\bigcup_{\alpha} G_{\alpha}$ is open.\label{th_110_a}
    
    \item For any collection $\{F_{\alpha}\}$ of closed sets, $\bigcap_{\alpha} F_{\alpha}$ is closed.\label{th_110_b}
    
    \item For any finite collection $G_1, \cdots, G_n$ of open sets, $\bigcap^n_{i=1} G_i$ is open.\label{th_110_c}
    
    \item For any finite collection $F_1, \cdots, F_n$ of closed sets, $\bigcup^n_{i=1} F_i$ is closed.\label{th_110_d}
\end{enumerate}
\end{theorem}
\begin{proof}
Let $G = \bigcup_{\alpha} G_{\alpha}$. If $x \in G$, then $x \in G_{\alpha}$ for some $\alpha$. Since $G_{\alpha}$ is open, then $x$ is an interior point of $G_{\alpha}$, and of course an interior point of $G$. Hence $G$ is open, and this proves \ref{th_110_a}.

By Theorem \ref{th_18}, 
\begin{align}\label{th_110_equ_1}
    \left(\bigcap_{\alpha} F_{\alpha}\right)^c = \bigcup_{\alpha} \left(F_{\alpha}^c\right),
\end{align}
and $F_{\alpha}^c$ is open by Theorem \ref{th_19}. Also, \ref{th_110_a} implies that \eqref{th_110_equ_1} is open and thus $\bigcap_{\alpha} F_{\alpha}$ is closed.

Now let $H = \bigcap^n_{i=1} G_i$, for any $x \in H$, there is neighborhoods $B(x,r_i)$ of $x$ such that each $B(x,r_i) \subset G_i, i = 1, \cdots, n$. Let $r = \min \{r_1, \cdots, r_n\}$, then clearly $B(x,r) \subset G_i$ for every $i = 1, \cdots, n$, and hence $B(x,r) \subset G$. Thus $G$ is open.

Similarly, \ref{th_110_d} follows from \ref{th_110_c} by
\begin{align*}
    \left(\bigcup^n_{i=1} F_{\alpha}\right)^c = \bigcap^n_{i=1} \left(F_{\alpha}^c\right)
\end{align*}
\end{proof}

\begin{remark}
In \ref{th_110_c} and \ref{th_110_d}, the finiteness of the union and intersection is required. For example, for $n = 1,2,3,\cdots$, let 
\begin{align*}
    G_n = \left(- \frac{1}{n}, \frac{1}{n}\right), 
\end{align*}
then $G = \bigcap^\infty_{n=1} G_n = \{0\}$, which is closed.  

Similarly, for $n = 1,2,3,\cdots$, let  
\begin{align*}
    F_n = \left[\frac{1}{n}, 1 - \frac{1}{n}\right],
\end{align*}
then $F = \bigcup^\infty_{n=1} F_n = (0,1)$, which is not closed.
\end{remark}

\medskip

\begin{definition}
Given $A \subset X$, the interior of the set $A$ is defined as the set of all points $x \in A$ that has a neighborhood contained in $A$, that is
\begin{align*}
    \operatorname{int}(A) = \{x \in A \,:\, \exists \, r > 0, B(x,r) \subset A\}.
\end{align*}
\end{definition}

\medskip

\begin{theorem}
The interior $\operatorname{int}(A)$ is always open and it is the largest open set contained in $A$ in the sense that if $G \subset A$ is open, then $G \subset \operatorname{int}(A)$.
\end{theorem}
\begin{proof}
Clearly, $\operatorname{int}(A)$ is open. Indeed, if $x \in \operatorname{int}(A)$, then there is $r > 0$ such that $B(x,r) \subset A$. Now consider any $y \in B(x,r/2)$, then $d(x,y) < r/2$. Then for any $z \in B(y, r/2)$, we have
\begin{align*}
    d(x,z) \leq d(x,y) + d(y,z) < r,
\end{align*}
which implies that every point in $B(x,r/2)$ is an interior point of $A$. Hence, there is a neighborhood $B(x,r/2)$ of $x$ such that $B(x,r/2) \subset \operatorname{int}(A)$, thus $\operatorname{int}(A)$ is open.

Now, let $G \subset A$ be any open set. Then for any $x \in G$, there is $r > 0$ such that $B(x,r) \subset G \subset A$, hence $x \in \operatorname{int}(A)$. Thus $G \subset \operatorname{int}(A)$.
\end{proof}

\medskip

\begin{definition}
If $X$ is a metric space, if $E \subset X$ and $E'$ denotes the set of all limit points of $E$, then the closure of $E$ is the set $\overline{E} = E \cup E'$.
\end{definition}

\medskip

\begin{theorem}
If $X$ is a metric space and $E \subset X$, then
\begin{enumerate}[label=(\alph*)]
    \item $\overline{E}$ is closed; \label{th_112_a}
    
    \item $E = \overline{E}$ if and only if $E$ is closed; \label{th_112_b}
    
    \item $\overline{E} \subset F$ for every closed set $F \subset X$ that contains $E$. \label{th_112_c}
\end{enumerate}
By \ref{th_112_a} and \ref{th_112_c}, $\overline{E}$ is the smallest closed subset of $X$ that contains $E$.
\end{theorem}
\begin{proof}
~\begin{enumerate}[label=(\alph*)]
    \item If $x \notin \overline{E}$, then $x$ is neither a point of $E$ nor a limit point of $E$. Then there is a neighborhood $B$ of $x$ such that $B \cap E = \emptyset$. Hence, $\overline{E}^c$ is open, and thus $\overline{E}$ is closed; 
    
    \item If $E = \overline{E}$, then \ref{th_112_a} implies $E$ is closed. If $E$ is closed, then $E' \subset E$, hence $\overline{E} = E \cup E' = E$;
    
    \item If $F$ is closed and $E \subset F$, then $F' \subset F$, hence $E' \subset F'$. Thus $\overline{E} \subset F$.
\end{enumerate}
\end{proof}

\medskip

\begin{theorem}\label{th_113}
Let $E$ be a nonempty set of real numbers which is bounded above. Let $y = \sup E$.\footnote{Recall the {\em least upper bound} or the {\em supremum}. Suppose $S$ is an ordered set, $E \subset S$ and $E$ is bounded above. Then the supremum $\alpha \in S$ satisfies the following properties: \begin{enumerate*}
    \item[(i)] $\alpha$ is an upper bound of $E$,
    \item[(ii)] If $\gamma < \alpha$, then $\gamma$ is not an upper bound of $E$.
\end{enumerate*} And we write $\alpha = \sup E$. The {\em greatest lower bound} or the {\em infimum} is defined in a similar way.}Then $y \in \overline{E}$. Hence $y \in E$ if $E$ is closed.
\end{theorem}
\begin{proof}
If $y \in E$, then $y \in \overline{E}$. Assume that $y \notin E$, for every $\varepsilon > 0$, there exists $x \in E$ such that $y - \varepsilon < x < y$, otherwise $y - \varepsilon$ is an upper bound of $E$. Hence, $y$ is a limit point of $E$, and thus $y \in \overline{E}$.
\end{proof}

\begin{remark}
Suppose $E \subset Y \subset X$. We say that $E$ is {\em open relative} to $Y$ if to each $x \in E$, there is $r > 0$ such that $y \in E$ whenever $d(x,y) < r$ and $y \in Y$. We talk about this since a set may be open relative to $Y$ without being an open subset of $X$. For example, let $E = (a,b)$, $a < b$ and $a,b \in \mathbb{R}$, $Y = \mathbb{R}$ and $X = \mathbb{R}^2$, then $(a,b)$ is an open subset of $\mathbb{R}$, but not an open subset of $\mathbb{R}^2$.
\end{remark}

\medskip

\begin{theorem}
Suppose $Y \subset X$. A subset $E$ of $Y$ is open relative to $Y$ if and only if $E = Y \cap G$ for some open subset $G$ of $X$.
\end{theorem}
\begin{proof}
Suppose $E$ is open relative to $Y$. For each $x \in E$, there is $r_x > 0$ such that $y \in E$ if $d(x,y) < r_x$ and $y \in Y$. Let $G_x = \{y \in X \,:\, d(x,y) < r_x\}$, and define
\begin{align*}
    G = \bigcup_{x \in E} G_x.
\end{align*}
Then $G$ is an open subset of $X$ by Theorem \ref{th_110}. Clearly, $E \subset G \cap Y$. Also, $G_x \cap Y \subset E$ for every $x \in E$, so that $G \cap Y \subset E$. Hence, $E = G \cap Y$. 

Conversely, if $G$ is open in $X$ and $E = G \cap Y$, every $x \in E$ has a neighborhood $B_x \subset G$. Then, $B_x \cap Y \subset E$, and thus $E$ is open relative to $Y$.
\end{proof}

\medskip


\section{Compact Sets}

\begin{definition}
An open cover of a set $E$ in a metric space $X$ is a collection $\{G_{\alpha}\}$ of open subsets of $X$ such that $E \subset \bigcup_{\alpha} G_{\alpha}$.
\end{definition}

\medskip

\begin{definition}
A subset $K$ of a metric space $X$ is said to be compact if every open cover of $K$ contains a finite subcover. More explicitly, this requirement is that if $\{G_{\alpha}\}$ is an open cover of $K$, then there are finitely many $\alpha_1, \cdots, \alpha_n$ such that 
\begin{align*}
    K \subset \bigcup^n_{i=1} G_{\alpha_i}.
\end{align*}
\end{definition}

\begin{remark}
With the familiarity of continuity, compactness can also be defined as follows: $K \subset X$ is compact if every sequence in $K$ has subsequence converging to a point in $K$. We will talk more about this later.
\end{remark}

\medskip

\begin{theorem}\label{th_115}
Compact subsets of metric space are closed.
\end{theorem}
\begin{proof}
Let $K$ be a compact subset of a metric space $X$. It suffices to prove that $K^c$ is open.

Suppose that $x \in K$ and $y \in K$, and let $B(x,r_x)$ and $B(y,r_x)$ be neighborhoods of $x$ and $y$, respectively, of radius $r_x$ less than $d(x,y)/2$. Clearly, $\bigcup_x B(x,r_x)$ is an open cover. Since $K$ is compact, then there is a finite subcover $\{B\left(x_i, r_{x_i}\right)\}^n_{i=1}$ such that 
\begin{align*}
    K \subset \bigcup^n_{i=1} B\left(x_i, r_{x_i}\right) = W.
\end{align*}
Now let $r_y = \min \{r_{x_1}, \cdots, r_{x_n}\}$, then $B(y,r_y) \cap W = \emptyset$, and hence $B(y,r_y) \subset K^c$. Hence $y$ is an interior point of $K^c$, and the rest follows.
\end{proof}

\medskip

\begin{theorem}\label{th_116}
Closed subsets of compact sets are compact.
\end{theorem}
\begin{proof}
Suppose $F \subset K \subset X$, $F$ is closed (relative to $X$), and $K$ is compact. Let $\{B_{\alpha}\}$ be an open cover of $F$. If $F^c \cap \bigcup_{\alpha}B_{\alpha} = \emptyset$, then we obtain an open cover $\Omega$ of $K$. Since $K$ is compact, there is a finite subcollection $\Phi$ of $\Omega$ which covers $K$, and hence $F$. If $F^c$ is a member of $\Phi$, then we can remove it from $\Phi$ and still obtain an open cover of $F$. Thus a finite subcollection of $\{B_{\alpha}\}$ covers $F$.
\end{proof}

\medskip

\begin{theorem}
If $F$ is closed and $K$ is compact, then $F \cap K$ is compact.
\end{theorem}
\begin{proof}
Theorems \ref{th_110} \ref{th_110_b} and \ref{th_115} show that $F \cap K$ is closed, since $F \cap K \subset K$, and Theorem \ref{th_116} shows that $F \cap K$ is compact.
\end{proof}

\medskip

\begin{theorem}\label{th_118}
If $\{K_{\alpha}\}$ is a collection of compact subsets of a metric space $X$ such that the intersection of every finite subcollection of $\{K_{\alpha}\}$ is nonempty, then $\cap_{\alpha} K_{\alpha}$ is nonempty.
\end{theorem}
\begin{proof}
Fix $K_1$ and let $G_{\alpha} = K_{\alpha}^c$. Assume no point of $K_1$ belongs to every $K_{\alpha}$. Then the sets $G_{\alpha}$ form an open cover of $K_1$. Since $K_1$ is compact, there are finitely many $\alpha_1, \cdots, \alpha_n$ such that
\begin{align*}
    K_1 \subset \bigcup^n_{i=1} G_{\alpha_i}.
\end{align*}
However, this implies
\begin{align*}
    K_1 \cap \left(\bigcup^n_{i=1} G_{\alpha_i}\right)^c = K_1 \cap \left(\bigcap^n_{i=1} G_{\alpha_i}^c\right) = K_1 \cap \left(\bigcap^n_{i=1} K_{\alpha_i}\right) = \emptyset,
\end{align*}
which is a contradiction.
\end{proof}

\medskip

\begin{corollary}\label{coro_118_1}
If $\{K_n\}$ is a sequence of nonempty compact sets such that $K_n \supset K_{n+1}, n = 1,2,3,\cdots$, then $\bigcap^\infty_{n=1} K_n$ is nonempty.
\end{corollary}

\medskip

\begin{theorem}\label{th_119}
If $E$ is an infinite subset of a compact set $K$, then $E$ has a limit point in $K$.
\end{theorem}
\begin{proof}
If no point in $K$ is a limit point of $E$, then each $x \in K$ would have a neighborhood $B(x,r)$ which contains at most one point of $E$, denoted by $y$, if $y \in E$. Then no finite open cover can cover $E$ and the same is true for $K$, since $E \subset K$, which is a contradiction.
\end{proof}

\medskip

\begin{theorem}\label{th_120}
If $\{I_n\}$ is a sequence of intervals in $\mathbb{R}$ such that $I_n \supset I_{n+1}, n = 1,2,3,\cdots$, then $\bigcap^\infty_{n=1} I_n$ is nonempty.
\end{theorem}
\begin{proof}
If $I_n = [a_n, b_n]$, let $E$ be the set of all $a_n$, then $E$ is nonempty and bounded above by $b_1$. Let $x = \sup E$. If $m$ and $n$ are positive integers, then 
\begin{align*}
    a_n \leq a_{n+m} \leq b_{n+m} \leq b_n,
\end{align*}
and hence $x \leq b_n$ for every $n$. Since $a_n \leq x$ for all $n$, it is clear that $x \in I_n$ for every $n$.
\end{proof}

\medskip

\begin{definition}
Let $a_n, b_n \in \mathbb{R}$ and $a_n < b_n$ for $n = 1,2,3,\cdots$, the set of all points $x = (x_1, \cdots, x_k) \in \mathbb{R}^k$ whose coordinates satisfy the inequalities $a_n \leq x_n \leq b_n, n = 1,2,3,\cdots$ is called a $k$-cell.
\end{definition}

\medskip

\begin{theorem}\label{th_121}
Let $k$ be a positive integer. If $\{I_n\}$ is a sequence of $k$-cells such that $I_n \supset I_{n+1}, n = 1,2,3,\cdots$, then $\bigcap^\infty_{n=1} I_n$ is nonempty.
\end{theorem}
\begin{proof}
For $x = (x_1, \cdots, x_k) \in \mathbb{R}^k$, let $I_n = \{x \,:\, a_{n,j} \leq x_j \leq b_{n,j}\}, 1 \leq j \leq k, n = 1,2,3\cdots$, and let $I_{n,j} = [a_{n,j}, b_{n,j}]$. For each $j$, the sequence $\{I_{n,j}\}$ satisfies the hypotheses of Theorem \ref{th_120}, hence there are real numbers $x_j^*, 1 \leq j \leq k$ such that $a_{n,j} \leq x_j^* \leq b_{n,j}$ for $1 \leq j \leq k, n = 1,2,3\cdots$. Now let $x^* = (x_1^*,\cdots, x_k^*)$, then it is clear that $x^* \in I_n$ for all $n = 1,2,3,\cdots$.
\end{proof}

\medskip

\begin{theorem}\label{th_122}
Every $k$-cell is compact.
\end{theorem}
\begin{proof}
Let $I$ be a $k$-cell, which consists of all points $x = (x_1, \cdots, x_k)$ such that $a_j \leq x_j \leq b_j$ for $1 \leq j \leq k$. Let
\begin{align*}
    \delta = \left(\sum^k_{j=1} (b_j - a_j)^2\right)^{1/2}.
\end{align*}
Then $\left|x - y\right| \leq \delta$ for all $x,y \in I$.

Suppose to the contrary that there exists an open cover $\{G_{\alpha}\}$ of $I$ which contains no finite subcover of $I$. Let $c_j = (a_j + b_j)/2$. The intervals $[a_j, c_j]$ determines $2^k$ $k$-cells $Q_i$ whose union is $I$. By assumption, at least one of these $Q_i$, say $I_1$, cannot be covered by any finite subcover of $\{G_{\alpha}\}$. We now divide $I_1$ and continue this process, and we obtain a sequence $\{I_n\}$ with the following properties:
\begin{enumerate}[label=(\alph*)]
    \item $I \supset I_1 \supset I_2 \supset I_3 \cdots$; \label{th_122_a}
    
    \item $I_n$ is not covered by any finite subcollection of $\{G_{\alpha}\}$; \label{th_122_b}
    
    \item if $x,y \in I_n$, $\left|x - y\right| \leq 2^{-n} \delta$. \label{th_122_c}
\end{enumerate}
By \ref{th_122_a} and Theorem \ref{th_121}, there is $x^*$ which belongs to every $I_n$. For some $\alpha$, $x^* \in G_{\alpha}$. Since $G_{\alpha}$ is open, there is $r > 0$ such that $\left|y - x^*\right| < r$ implies that $y \in G_{\alpha}$. Letting $n$ large enough such  that $2^{-n}\delta < r$, then \ref{th_122_c} implies that $I_n \subset G_{\alpha}$, which is a contradiction to \ref{th_122_b}.
\end{proof}

\medskip

\begin{theorem}[Heine-Borel]\label{th_123}
If a set $E \subset \mathbb{R}^n$ has one of the following three properties, then it has the other two:
\begin{enumerate}[label=(\alph*)]
    \item $E$ is closed and bounded. \label{th_123_a}
    
    \item $E$ is compact. \label{th_123_b}
    
    \item Every infinite subset of $E$ has a limit point in $E$. \label{th_123_c}
\end{enumerate}
The equivalence of \ref{th_123_a} and \ref{th_123_b} is known as the Heine-Borel theorem.
\end{theorem}
\begin{proof}
If \ref{th_123_a} holds, then $E \subset I$ for some $k$-cell $I$, and hence \ref{th_123_b} follows from Theorems \ref{th_122} and \ref{th_116}. Also, \ref{th_123_b} implies \ref{th_123_c} by Theorem \ref{th_119}. It remains to show that \ref{th_123_c} implies \ref{th_123_a}.

Suppose to the contrary that $E$ is not not bounded, then $E$ contains points $x_n$ with $\left|x_n\right| > n, n = 1,2,3,\cdots$. The set consisting of all $x_n$ is infinite and clearly has no limit point in $\mathbb{R}^n$, hence has no limit point in $E$. Hence, \ref{th_123_c} implies that $E$ is bounded.

If $E$ is not closed, then there is a point $x \in \mathbb{R}^n$ which is a limit of $E$ but $x \notin E$. For $n = 1,2,3,\cdots$, there are points $x_n \in E$ such that $\left|x_n - x\right| < 1/n$. Let $S = \{x_n\}$, then $S$ is infinite and $x$ is a limit point of $S$. Also, $x$ is the only limit point of $S$. Indeed, for any $y \in \mathbb{R}^n$, $y \neq x$, then for all but finitely many $n$,
\begin{align*}
    \left|x_n - y\right| & \geq \left|x_n - x\right| - \left|x - y\right| \geq \left|x - y\right| - \frac{1}{n} \geq \frac{1}{2} \left|x - y\right|,
\end{align*}
which implies that $y$ is not a limit point of $S$. This is a contradiction to \ref{th_123_c} since $S \subset E$.
\end{proof}

\begin{remark}
Note that \ref{th_123_b} and \ref{th_123_c} are equivalent in any metric space but in general, \ref{th_123_a} does not implies \ref{th_123_b} and \ref{th_123_c}. More details will be discussed later.
\end{remark}

\medskip

\begin{theorem}[Weierstrass]\label{}
Every bounded infinite subset of $\mathbb{R}^n$ has a limit point in $\mathbb{R}^n$.
\end{theorem}
\begin{proof}
Suppose that $E \subset \mathbb{R}^n$ is bounded, then $E$ is a subset of a $k$-cell $I \subset \mathbb{R}^n$. By Theorem \ref{th_122}, $I$ is compact, and Theorem \ref{th_123} implies that $E$ has a limit point in $I$.
\end{proof}

\medskip



\section{Perfect Sets}

\begin{theorem}\label{th_125}
Let $P$ be a nonempty perfect set in $\mathbb{R}^n$, then $P$ is uncountable.
\end{theorem}
\begin{proof}
Since $P$ has limit points, $P$ is infinite. Suppose $P$ is countable, then $P$ can be written as $\{x_1, x_2, x_3, \cdots\}$. Let $V_1$ be any neighborhood of $x_1$. Suppose $V_n$ has been constructed, so that $V_n \cap P \neq \emptyset$. Since every point of $P$ is a limit of $P$, there is a neighborhood $V_{n+1}$ of $x$ such that \begin{enumerate*}[label=(\roman*)]
    \item $\overline{V}_{n+1} \subset V_n$,
    \item $x_n \notin \overline{V}_{n+1}$,
    \item $V_{n+1} \cap P \neq \emptyset$.\label{th_125_iii}
\end{enumerate*}
By \ref{th_125_iii}, $V_{n+1}$ satisfies our induction hypothesis, and the process can proceed. 

Let $K_n = \overline{V}_{n} \cap P$. Since $\overline{V}_{n}$ is closed and bounded, $\overline{V}_{n}$ is compact. Since $x_n \notin K_{n+1}$, no point of $P$ lies in $\bigcap^\infty_{n=1} K_n$. Since $K_n \subset P$, this implies $\bigcap^\infty_{n=1} K_n = \emptyset$. However, each $K_n \neq \emptyset$ and $K_n \supset K_{n+1}$, and by Corollary \ref{coro_118_1}, $\bigcap^\infty_{n=1} K_n$ is nonempty, thus this is a contradiction.
\end{proof}

\medskip

\begin{corollary}
Every interval $[a,b], a < b$ is uncountable. In particular, the set $\mathbb{R}$ of all real numbers is uncountable.
\end{corollary}

\medskip

\begin{example}[The Cantor set]
The Cantor set shows that there exist perfect sets in $\mathbb{R}$ which contains no segment, and it is constructed as follows:
\begin{enumerate}[label=(\roman*)]
    \item Let $\mathcal{C}_1$ be the interval $[0,1]$;
    
    \item Remove the segment $\left(1/3,2/3\right)$, and let $\mathcal{C}_2$ be the union of the interval of $\left[0,1/3\right],\left[1/3,1\right]$;
    
    \item Remove the middle thirds of these two intervals and let $\mathcal{C}_3$ be the union of $\left[0,1/9\right]$, $\left[2/9,1/3\right]$, $\left[6/9,7/9\right]$ and $\left[8/9,1\right]$;
    
    \item Continue this way and we can get a sequence of compact sets $\mathcal{C}_n$, such that $\mathcal{C}_1\supset \mathcal{C}_2\supset \mathcal{C}_3\supset\cdots$ and $\mathcal{C}_n$ is the union of $2^n$ intervals with length $3^{-n}$.
\end{enumerate}
Then the {\em Cantor set} is defined as 
\begin{align*}
    \mathcal{C} = \bigcap^\infty_{i=1}\mathcal{C}_i.
\end{align*}
$\mathcal{C}$ is clearly compact and by Theorem \ref{th_118}, $\mathcal{C}$ is nonempty.

No segment of the form
\begin{align}\label{exam_12_equ1}
    \left(\frac{3k+1}{3^n}, \frac{3k+2}{3^n}\right),
\end{align}
where $k,n$ are positive integers, has a point in common with $\mathcal{C}$. Since every segment $(\alpha, \beta)$ contains a segment of the form \eqref{exam_12_equ1}, if 
\begin{align*}
    3^{-m} < \frac{\beta - \alpha}{6},
\end{align*}
$\mathcal{C}$ contains no segment.

To show that $\mathcal{C}$ is perfect, it is enough to show that $\mathcal{C}$ contains no isolated point. Let $x \in \mathcal{C}$, and let $S$ be any segment containing $x$. Let $I_n$ be the interval of $\mathcal{C}_n$ which contains $x$. Let $n$ large enough such that $I_n \subset S$. Let $x_n$ be an endpoint of $I_n$ such that $x_n \neq x$. It is clear that $x_n \in \mathcal{C}$ and hence $x$ is a limit point of $\mathbb{C}$. Thus $\mathbb{C}$ is perfect, and hence uncountable. 

One of the most interesting facts about the Cantor set is that it provides an example as an uncountable set of measure zero.
\end{example}

\medskip

Now we provide another approach to proving that the Cantor set is uncountable.

\medskip

\begin{proof} We use Cantor's first proof of uncountability. Let $\mathcal{C}$ be the cantor set, and let $E = \{u_1,u_2,\cdots\}$ be a countable set. We construct a point of $\mathcal{C}$ but not in $E$. First, $\mathcal{C} \subseteq [0,1/3] \cup [2/3,1]$, and point $u_1$ does not belong to both of those intervals, so there is an interval $I_1$of length $1/3$ (one of the ones in the first stage of the construction of the Cantor set) with $u_1 \notin I_1$. Now when we remove the middle third of $I_1$ we get two intervals of length $1/3^2$. As before, $u_2$ does not belong to both of these intervals, so there is an interval $I_2 \subset I_1$ of length $1/3^2$ (one of the ones in the second stage of the construction of the Cantor set) with $u_2 \notin I_2$. Continue in this way to get $I_1 \supset I_2 \supset I_3 \supset \cdots \supset I_k \supset \cdots$ where $I_k$ has length $1/3^k$ (one of the intervals in the $k$th stage of the construction of the Cantor set) and $u_k \notin I_k$. Finally, we get a point $x \in \bigcap^\infty_{k=1} I_k$ where $x \in \mathcal{C}$ but $x \neq u_k$ for all $k$.
\end{proof}

\medskip



\section{Connected Sets}

\begin{definition}
Let $X$ be a metric space, and $A \subset X$. $A$ is said to be disconnected if there exist open sets $U$ and $V$ in $X$ which satisfy the following properties:
\begin{enumerate}[label=(\alph*)]
    \item $A \subset U \cup U$;
    
    \item $A \cap U \neq \emptyset$ and $A \cap V \neq \emptyset$;
    
    \item $A \cap (U \cap V) = \emptyset$.
\end{enumerate}
Moreover, $A$ is called connected if it is not disconnected.
\end{definition}

\medskip

\begin{proposition}
Prove that the space $X$ is connected if and only if the only subsets of $X$, which are open and closed at the same time, are $\emptyset$ and $X$ itself.
\end{proposition}
\begin{proof}
If $X$ is connected, we need to show that if $E \subset X$ is open and closed, then $E = \emptyset$ or $E = X$. Suppose to the contrary that there is a set $E \subset X$ such that $E \neq \emptyset, E \neq X$ and $E$ is open and closed at the same time. Since $E$ is closed, then $E^c$ is open, and hence $X \subset (E \cup E^c)$. Also, $X \cap E \neq \emptyset$, $X \cap E^c \neq \emptyset$, and $X \cap (E \cap E^c) = \emptyset$. Hence $X$ is disconnected, which is a contradiction.

Conversely, suppose that $X$ is disconnected, then there are two open sets $U$ and $V$ in $X$ such that $X = U \cup V$, $X \cap U \neq \emptyset$, $X \cap V \neq \emptyset$ and $X \cap (U \cap V) = \emptyset$. Hence, $U \neq \emptyset$. Also, $U$ is also closed since $V = U^c$ is open. Thus $U$ is open and closed at the same time, which is a contradiction.
\end{proof}

\medskip

Next, we talk about another definition of connected sets.

\medskip

\begin{definition}
Two subsets $A$ and $B$ of a metric space $X$ are said to be separated if $A \cap \overline{B} = \emptyset$ and $\overline{A} \cap B = \emptyset$. A set $E \subset X$ is said to be connected if $E$ is not a union of two nonempty separated sets (this definition is equivalent to the previous one).
\end{definition}

\begin{remark}
Separated sets are of course disjoint, but disjoint sets need not be separated. For example, the intervals $[0,1]$ and $(1,2)$ are not separated, since $1$ is a limit point of $(1,2)$. However, $(0,1)$ and $(1,2)$ are separated.
\end{remark}

\medskip

\begin{theorem}
A subset $E \subset \mathbb{R}$ is connected if and only if it has the following property: If $x,y \in E$ and $x < z < y$, then $z \in E$.
\end{theorem}
\begin{proof}
Suppose there exist $x, y \in E$ and $z \in (x,y)$, but $z \notin E$. Then $E = U \cup V$, where 
\begin{align*}
    U = (-\infty, z) \cap E,\qquad V = (z, \infty) \cap E.
\end{align*}
Since $x \in U$ and $y \in V$, $U$ and $V$ are nonempty. Also, since $U \cap V = \emptyset$, $E \cap (U \cap V) = \emptyset$, and hence $E$ is disconnected, which is a contradiction.

Conversely, suppose that $E$ is disconnected. Then there are nonempty separated sets $U$ and $V$ such that $E = U \cup V$. Let $x \in U$ and $y \in V$, and without loss of generality we assume that $x < y$. Let
\begin{align*}
    z = \sup (U \cap [x,y]).
\end{align*}
By Theorem \ref{th_113}, $z \in \overline{U}$, hence $z \notin V$. In particular, $x \leq z < y$. If $z \notin U$, it follows that $x < z < y$ and $z \notin E$. If $z \in U$, then $z \notin \overline{V}$, hence there exists $z_1$ such that $z < z_1 < y$ and $z_1 \notin E$, which is a contradiction.
\end{proof}










\newpage

\chapter{Numerical Sequences and Series}

\section{Convergent Sequences}

\begin{definition}
A sequence $\{x_n\}$ in a metric space $X$ is said to converge if there is a point $x \in X$ with the following property: For every $\varepsilon > 0$, there is an integer $N$ such that for all $n \geq N$, $d(x_n,x) < \varepsilon$. (Here $d$ denotes the distance in $X$.)

In this case, we also say that $\{x_n\}$ converges to $x$, or $x$ is the limit of $\{x_n\}$, and we write $x_n \to x$, or
\begin{align*}
    \lim_{n\to\infty} x_n = x.
\end{align*}

If $\{x_n\}$ does not converge, we say it diverge\footnote{The formal definition of being divergent can be expressed as: If for every $x \in X$, there exists $\varepsilon > 0$ such that for all $N \in \mathbb{N}$, there is $n \geq N$, $d(x_n,x) > \varepsilon$. Also, we say a sequence $\{x_n\}$ diverges to $+\infty$ if for all $M > 0$, there is an integer $N > 0$ such that for all $n \geq N$, $x_n > M$. Then we write $\lim_{n\to\infty} x_n = + \infty$. Similarly, we could define a sequence that diverges to $-\infty$.}.
\end{definition}

\medskip

\begin{theorem}\label{th_21}
Let $\{x_n\}$ be a sequence in a metric space $X$.
\begin{enumerate}[label=(\alph*)]
    \item $\{x_n\}$ converges to $x \in X$ if and only if every neighborhood of $x$ contains $x_n$ for all but finitely many $n$. \label{th_21_a}
    
    \item If $x \in X$, $x' \in X$ and if $\{x_n\}$ converges to $x$ and to $x'$, then $x = x'$. \label{th_21_b}
    
    \item If $\{x_n\}$ converges, then $\{x_n\}$ is bounded. \label{th_21_c}
    
    \item If $E \subset X$ and $x$ is a limit point of $E$, then there is a sequence $\{x_n\}$ in $E$ such that $\lim_{n\to\infty} x_n = x$. \label{th_21_d}
\end{enumerate}
\end{theorem}
\begin{proof}
~\begin{enumerate}[label=(\alph*)]
    \item Suppose $x_n \to x$ and let $V$ be a neighborhood of $x$. For some $\varepsilon > 0$, $d(x,y) < \varepsilon, y \in X$ implies that $y \in V$. Corresponding to this $\varepsilon$, there is $N > 0$ such that for all $n \geq N$, $d(x_n,x) < \varepsilon$. Hence for all $n \geq N$, $x_n \in V$.
    
    Conversely, suppose every neighborhood of $x$ contains all but finitely many $x_n$. Fix $\varepsilon > 0$, and let $V$ be the set of all $y \in X$ such that $d(x,y) < \varepsilon$. By assumption, there is $N > 0$ which depends on $V$, such that for $n \geq N$, $x_n \in V$. Hence $d(x_n, x) < \varepsilon$ for $n \geq N$, which implies that $x_n \to x$.
    
    \item Let $\varepsilon > 0$ be given, then there are integers $N, N'$ such that for all $n \geq N$, $d(x_n,x) < \varepsilon/2$ and for all $n \geq N'$, $d(x_n,x') < \varepsilon/2$. Hence if $n \geq \max \{N, N'\}$, we have
    \begin{align*}
        d(x,x') \leq d(x,x_n) + d(x_n,x') < \varepsilon.
    \end{align*}
    Since $\varepsilon$ is arbitrary, we conclude that $d(x,x') = 0$.
    
    \item Let $\varepsilon = 1$, then there is an integer $N$ such that for all $n \geq N$, $d(x_n, x) < 1$. Let \begin{align*}
        r = \max\{1, d(x_1,x), \cdots, d(x_N,x)\},
    \end{align*}
    then $d(x_n,x) \leq r$ for all $n = 1,2,3,\cdots$.
    
    \item Since $x$ is a limit point of $E$, then for neighborhood of $x$ with radius $1/n, n = 1,2,3,\cdots$, there is a point $x_n \in E$ such that $d(x_n,x) < 1/n$. For any $\varepsilon > 0$, let $N > 1/\varepsilon$, then for all $n \geq N$, $d(x_n, x) < \varepsilon$, which implies $x_n \to x$.
\end{enumerate}
\end{proof}

\medskip

Next, we talk about some important properties about convergent sequences.

\medskip

\begin{theorem}\label{th_22}
Suppose $\{a_n\}$ and $\{b_n\}$ are two complex sequences, and $\lim_{n\to\infty} a_n = a$ and $\lim_{n\to\infty} b_n = b$. Then,
\begin{enumerate}[label=(\alph*)]
    \item $\lim_{n\to\infty} (a_n + b_n) = a + b$; \label{th_22_a}
    
    \item $\lim_{n\to\infty} c a_n = ca$, $\lim_{n\to\infty} (c + a_n) = c + a_n$, for any $c \in \mathbb{R}$; \label{th_22_b}
    
    \item $\lim_{n\to\infty} a_n b_n = ab$; \label{th_22_c}
    
    \item $\lim_{n\to\infty} a_n/b_n = a/b$, provided $b_n \neq 0$ for all $n$ and $b \neq 0$. \label{th_22_d}
\end{enumerate}
\end{theorem}
\begin{proof}
~\begin{enumerate}[label=(\alph*)]
    \item and \ref{th_22_b} are trivial.
    
    \setcounter{enumi}{2}
    \item For every $\varepsilon > 0$, there are integers $N_1$ and $N_2$ such that for all $n \geq N_1$, $\left|a_n - a\right| < \sqrt{\varepsilon}$ and for all $n \geq N_2$, $\left|b_n - b\right| < \sqrt{\varepsilon}$. Let $N = \max \{N_1, N_2\}$, for all $n \geq N$, we have
    \begin{align*}
        \left|(a_n - a)(b_n - b)\right| < \varepsilon,
    \end{align*}
    which implies that $\lim_{n\to\infty} (a_n - a)(b_n - b) = 0$. Also, by \ref{th_22_a} and \ref{th_22_b},
    \begin{align*}
        \lim_{n\to\infty} (a_nb_n - ab) = \lim_{n\to\infty} \left[(a_n - a)(b_n - b) + a(b_n - b) + b(a_n - a)\right] = 0.
    \end{align*}
    
    \item For $\varepsilon = \left|b\right|/2$, there is an integer $N_1$ such that for all $n \geq N_1$, $ \left|b_n - b\right| < \left|b\right|/2$, and hence $\left|b_n\right| > \left|b\right|/2$ for $n \geq N_1$. 
    
    Now for any $\varepsilon > 0$, there is an integer $N_2$ such that for all $n \geq N_2$, $\left|b_n - b\right| < \left|b\right|^2\varepsilon/2$. Let $N = \max \{N_1, N_2\}$, then for all $n \geq N$, 
    \begin{align*}
        \left|\frac{1}{b_n} - \frac{1}{b}\right| = \left|\frac{b_n - b}{b_n b}\right| < \frac{2}{\left|b\right|^2} \left|b_n - b\right| < \varepsilon.
    \end{align*}
    Hence, $\lim_{n\to\infty} 1/b_n = 1/b$, and by \ref{th_22_c}, $\lim_{n\to\infty} a_n/b_n = a/b$.
\end{enumerate}
\end{proof}

\medskip

Now we can discuss the sequences in $\mathbb{R}^k$.

\medskip

\begin{theorem}
~\begin{enumerate}[label=(\alph*)]
    \item Suppose $x_n \in \mathbb{R}^k, n = 1,2,3,\cdots$ and $x_k = (x_{1,n}, \cdots, x_{k,n})$. Then $\{x_n\}$ converges to $x = (x_1, \cdots, x_k)$ if and only if 
    \begin{align*}
        \lim_{n\to\infty} x_{j,n} = x_j, \quad 1 \leq j \leq k.
    \end{align*}
    
    \item Suppose $\{x_n\}, \{y_n\}$ are sequences in $\mathbb{R}^k$, $\{\beta_n\}$ is a sequence of real numbers, and $x_n \to x, y_n \to y, \beta_n \to \beta$. Then,
    \begin{align*}
        \lim_{n\to\infty} (x_n + y_n) = x + y, \quad \lim_{n\to\infty} x_n \cdot y_n = x \cdot y, \quad \lim_{n\to\infty} \beta_n x_n = \beta x.
    \end{align*}
\end{enumerate}
\end{theorem}
\begin{proof}
The proof is trivial and we skip it here.
\end{proof}

\medskip

Now we talk some important results and applications about convergent sequences.

\medskip

\begin{theorem}
If $a_n \leq b_n \leq c_n$ and $\lim_{n\to\infty} a_n = \lim_{n\to\infty} c_n = g \in \mathbb{R}$, then $\lim_{n\to\infty} b_n = g$.
\end{theorem}

\medskip

We now talk about some special sequences which will be used frequently.

\medskip

\begin{theorem}\label{th_25}
~\begin{enumerate}[label=(\alph*)]
    \item If $a > 0$, then $ \lim_{n\to\infty} \displaystyle \frac{1}{n^a} = 0$.\label{th_25_a}
    
    \item If $a > 0$, $\lim_{n\to\infty} \displaystyle \sqrt[n]{a} = 1$.\label{th_25_b}
    
    \item For $n \in \mathbb{N}$, $ \lim_{n\to\infty} \displaystyle \sqrt[n]{n} = 1$.\label{th_25_c}
    
    \item If $a > 0$ and $k$ is real, then $ \lim_{n\to\infty} \displaystyle \frac{n^k}{(1 + a)^n} = 0$.\label{th_25_d}
    
    \item If $a > 1$, then $\lim_{n\to\infty} a^{-n} = 0$. If $\left|a\right| < 1$, then $\lim_{n\to\infty} a^n = 0$.\label{th_25_e}
\end{enumerate}
\end{theorem}
\begin{proof}
~\begin{enumerate}[label=(\alph*)]
    \item For every $\varepsilon > 0$, let $N > (1/\varepsilon)^{1/a}$, then for all $n \geq N$, 
    \begin{align*}
        \frac{1}{n^a} < \frac{1}{\left[(1/\varepsilon)^{1/a}\right]^a} = \varepsilon.
    \end{align*}
    
    \item If $a > 1$, let $x_n = \sqrt[n]{a} - 1$. By the Binomial formula,\footnote{For $a, b \in \mathbb{R}$ and $n \in \mathbb{N}$, \begin{align*}
    (a + b)^n = \sum^n_{k=1} \binom{n}{k} a^{n-k}b^{k} = \sum^n_{k=1} \frac{n!}{k!(n-k)!} a^{n-k}b^{k}.
    \end{align*}}we have $a = (1 + x_n)^n \geq 1 + nx_x$, and hence
    \begin{align*}
        0 < x_n < \frac{a - 1}{n},
    \end{align*}
    since the right hand side converges to $0$ as $n \to \infty$, we could conclude that $\sqrt[n]{a} - 1 \to 0$, which implies $\lim_{n\to\infty} \sqrt[n]{a} = 1$.

    If $a = 1$, then $\sqrt[n]{a} = 1 \to 1$. If $0 < a < 1$, then $1/a > 1$ and hence $\sqrt[n]{1/a} \to 1$. Thus, by Theorem \ref{th_22} \ref{th_22_d},
    \begin{align*}
        \lim_{n\to\infty} \sqrt[n]{a} = \lim_{n\to\infty} \frac{1}{\sqrt[n]{1/a}} = 1.
    \end{align*}
    
    \item Let $x_n = \sqrt[n]{n} - 1$, then $x_n > 0$ and applying the Binomial formula implies that 
    \begin{align*}
        n = (1 + x_n)^n \geq \binom{n}{2} 1^{n-1} x_n^2 = \frac{n(n-1)}{2} x_n^2,
    \end{align*}
    and hence
    \begin{align*}
        0 \leq x_n \leq \sqrt{\frac{2}{n-1}}.
    \end{align*}
    Since both the left and right hand sides converges to $0$, we have $\sqrt[n]{n} - 1 \to 0$.
    
    \item Let $m$ be an integer such that $m > k$. For $n > 2m$, we have
    \begin{align*}
        (1 + a)^n > \binom{n}{m}a^m = \frac{n!}{m!(n-m)!} a^m > \frac{n^m a^m}{2^m m!}.
    \end{align*}
    Hence, 
    \begin{align*}
        0 < \frac{n^k}{(1 + a)^n} < \frac{2^m m!}{a^m} n^{k - m} \xrightarrow[]{n\to\infty} 0.
    \end{align*}
    
    \item If $a > 1$, the Binomial formula implies 
    \begin{align*}
        a^n = (1 + (a - 1))^n \geq 1 + n(a - 1),
    \end{align*}
    and hence
    \begin{align*}
        0 < a^{-n} \frac{1}{1 + n(a - 1)} \xrightarrow[]{n\to\infty} 0.
    \end{align*}
    
    If $a = 0$, then it is clear that $a^n = 0$ and if $0 < \left|a\right| < 1$, then $1/\left|a\right| > 1$. Hence,
    \begin{align*}
        \left|a^n - 0\right| = \left(\frac{1}{\left|a\right|}\right)^{-n} \xrightarrow[]{n\to\infty} 0.
    \end{align*}
\end{enumerate}
\end{proof}

\medskip

\begin{definition}
The extended real line $\overline{\mathbb{R}}$ is defined as $\overline{\mathbb{R}} = \mathbb{R} \cup \{-\infty, \infty\}$.
\end{definition}

\medskip

Some important results about convergent sequences will be shown below.

\medskip

\begin{theorem}\label{th_26}
If $\lim_{n\to\infty} a_n = a \in \overline{\mathbb{R}}$, then
\begin{align*}
    \lim_{n\to\infty} \frac{a_1 + a_2 + \cdots + a_n}{n} = a.
\end{align*}
\end{theorem}
\begin{proof}
Since $a_n \to a$, then for every $\varepsilon > 0$, there is an integer $N_1$ such that for all $n \geq N_1$, $\left|a_n - a\right| < \varepsilon$. Also, since $\{a_n\}$ is convergent, there is $M > 0$ such that $\left|a_n\right| \leq M$ for all $n$ and $\left|a\right| < M$. Now, let $N = \max \left\{2N_1 M/\varepsilon, N_1\right\}$, then for all $n \geq N$, we have
\begin{align*}
    \left|\frac{a_1 + \cdots + a_n}{n} - a\right| & = \left|\frac{(a_1-a) + \cdots + (a_n-a)}{n}\right| \\
    & \leq \left|\frac{(a_1-a) + \cdots + (a_{N_1}-a)}{n}\right| + \left|\frac{(a_{N_1+1}-a) + \cdots + (a_n-a)}{n}\right| \\
    & \leq \frac{2N_1M}{n} + \frac{n - N_1}{n} \varepsilon \leq 2 \varepsilon.
\end{align*}
\end{proof}

\medskip

\begin{theorem}\label{th_27}
If $\lim_{n\to\infty} a_n = a \in \overline{\mathbb{R}}$, and $a_n > 0$ for all $n$, then
\begin{align*}
    \lim_{n\to\infty} \sqrt[n]{a_1 a_2 \cdots a_n} = a.
\end{align*}
\end{theorem}
\begin{proof}
Since $a_n \to a$, then for every $\varepsilon > 0$, there is an integer $N$ such that for all $n \geq N$, $\left|a_n - a\right| < \varepsilon$. Hence, for any $n > N$, since $a_1 a_2 \cdots a_N (g+\varepsilon)^{-N} > 0$, by Theorem \ref{th_25} \ref{th_25_b}, we have
\begin{align*}
    \limsup_{n\to\infty} \sqrt[n]{a_1 a_2 \cdots a_n} & = \limsup_{n\to\infty} \sqrt[n]{a_1 a_2 \cdots a_N} \sqrt[n]{a_{N+1}\cdots a_n} \\
    & \leq \limsup_{n\to\infty} \sqrt[n]{a_1 a_2 \cdots a_N} \sqrt[n]{(g+\varepsilon)^{n-N}} \\
    & = \limsup_{n\to\infty} \sqrt[n]{a_1 a_2 \cdots a_N (g+\varepsilon)^{-N}} \sqrt[n]{(g+\varepsilon)^n} \\
    & = (g+\varepsilon) \cdot \limsup_{n\to\infty} \sqrt[n]{a_1 a_2 \cdots a_N (g+\varepsilon)^{-N}} = g+\varepsilon.
\end{align*}
Similarly, we can prove that $g - \varepsilon \leq \liminf_{n\to\infty} \sqrt[n]{a_1 a_2 \cdots a_n}$, and the result follows.
\end{proof}

\begin{remark}
We could only write $\limsup$ and $\liminf$ in the above theorem, since we do not know if the limit exists before we prove it. 
\end{remark}

\medskip

\begin{theorem}
If $a_n > 0$ for all $n$ and $\lim_{n\to\infty} a_{n+1}/a_n = a \in \overline{\mathbb{R}}$, then $\lim_{n\to\infty} \sqrt[n]{a_n} = a$.
\end{theorem}
\begin{proof}
Since $\lim_{n\to\infty} a_{n+1}/a_n = a$, then the following sequence also converges to $a$:
\begin{align*}
    a_1, \frac{a_2}{a_1}, \cdots, \frac{a_n}{a_{n-1}}, \cdots.
\end{align*}
Hence, Theorem \ref{th_27} implies
\begin{align*}
    \lim_{n\to\infty} \sqrt[n]{a_n} & = \lim_{n\to\infty} \sqrt[n]{a_1 \cdot \frac{a_2}{a_1} \cdots \frac{a_n}{a_{n-1}}} = a.
\end{align*}
\end{proof}

\medskip

\begin{theorem}[Stolz]\label{th_29}
Suppose that $\{x_n\}$ and $\{y_n\}$ are two sequences of real numbers that satisfy the following properties:
\begin{enumerate}[label=(\alph*)]
    \item there is $N$ such that $0 < y_n < y_{n+1}$ for all $n \geq N$;  \label{th_29_a}
    
    \item $\lim_{n\to\infty} y_n = \infty$. \label{th_29_b}
\end{enumerate}
If
\begin{align*}
    \lim_{n\to\infty} \frac{x_n - x_{n-1}}{y_n - y_{n-1}} = g \in \overline{\mathbb{R}},
\end{align*}
then 
\begin{align*}
    \lim_{n\to\infty} \frac{x_n}{y_n} = g.
\end{align*}
\end{theorem}
\begin{proof}
First we assume that $g$ is finite, that is $g \in \mathbb{R}$. For every $\varepsilon > 0$, there is an integer $N > 0$ such that for all $n \geq N$, 
\begin{align}\label{th_29_equ1}
    \left|\frac{x_{n+1} - x_n}{y_{n+1} - y_n} - g\right| < \frac{\varepsilon}{2}.
\end{align}
Note that if $b,d > 0$ and $a,b < c < d$, then
\begin{align}\label{th_29_equ2}
    \frac{a}{b} < \frac{a+c}{b+d} < \frac{c}{d}.
\end{align}

Applying \eqref{th_29_equ2} to \eqref{th_29_equ1} with induction implies that for all $n > N$ we have
\begin{align*}
    \frac{x_n - x_N}{y_n - y_N} = \frac{\sum^{n-1}_{i=N} (x_{i+1} - x_i)}{\sum^{n-1}_{i=N} (y_{i+1} - y_i)} \in \left(g - \frac{\varepsilon}{2}, g + \frac{\varepsilon}{2}\right),
\end{align*}
which is equivalent to that for all $n > N$,
\begin{align}
    \left|\frac{x_n - x_N}{y_n - y_N} - g\right| < \frac{\varepsilon}{2}.
\end{align}
Also, 
\begin{align*}
    \frac{x_n}{y_n} - g = \frac{x_N - gy_N}{y_n} + \left(1 - \frac{y_N}{y_n}\right) \left(\frac{x_n - x_N}{y_n - y_N} - g\right).
\end{align*}
Since $\left|1 - y_n/y_n\right| < 1$ for $n > N$, we have
\begin{align}
    \left|\frac{x_n}{y_n} - g\right|  \leq \left|\frac{x_N - gy_N}{y_n}\right| + \left|\frac{x_n - x_N}{y_n - y_N} - g\right|.
\end{align}

Let $N_1 > N$ be such that for all $n > N_1$,
\begin{align}
    \left|\frac{x_N - gy_N}{y_n}\right| < \frac{\varepsilon}{2},
\end{align}
and the existence of $N_1$ follows from \ref{th_29_b}. Hence for all $n > N_1$, we have
\begin{align*}
    \left|\frac{x_n}{y_n} - g\right| \leq \frac{\varepsilon}{2} + \frac{\varepsilon}{2} = \varepsilon.
\end{align*}

Now we consider the case $g = \infty$, then there is an integer $N_2 > 0$ such that for $n \geq N_2$,
\begin{align*}
    x_n - x_{n-1} > y_n - y_{n-1} > 0,
\end{align*}
and hence $\lim_{n\to\infty} x_n = \infty$. Since
\begin{align*}
    \lim_{n\to\infty} \frac{x_n - x_{n-1}}{y_n - y_{n-1}} = \infty,
\end{align*}
we have
\begin{align*}
    \lim_{n\to\infty} \frac{y_n - y_{n-1}}{x_n - x_{n-1}} = 0.
\end{align*}
And the assumptions of Stolz's theorem are satisfied and hence
\begin{align*}
    \lim_{n\to\infty} \frac{y_n}{x_n} = 0,
\end{align*}
and thus 
\begin{align*}
    \lim_{n\to\infty} \frac{x_n}{y_n} = \infty.
\end{align*}
\end{proof}

\medskip

The Stolz's theorem can be useful, and as an example, we use it as a second method to prove Theorem \ref{th_26}.

\medskip

\begin{proof}[Second Proof of Theorem \ref{th_26}]
Let $x_n = a_1 + a_2 + \cdots + a_n$, and $y_n = n$, then
\begin{align*}
    \lim_{n\to\infty} \frac{a_1 + a_2 + \cdots + a_n}{n} = \lim_{n\to\infty} \frac{x_n - x_{n-1}}{y_n - y_{n-1}} = \lim_{n\to\infty} a_n = a.
\end{align*}
\end{proof}

\medskip


\section{Subsequences}

\begin{definition}
Given a sequence $\{x_n\}$, consider a sequence $\{n_k\}$ of positive integers such that $n_1 < n_2 < n_3 < \cdots$, then the sequence $\{x_{n_k}\}$ is called a subsequence of $\{x_n\}$. If $\{x_{n_k}\}$ converges, its limit is called a subsequential limit of $\{x_n\}$.
\end{definition}


\medskip

\begin{theorem}\label{th_210}
~\begin{enumerate}[label=(\alph*)]
    \item If $\{x_n\}$ is a sequence in a compact metric space $X$, then some subsequence of $\{x_n\}$ converges to a point of $X$. \label{th_210_a}
    
    \item Every bounded sequence in $\mathbb{R}^n$ contains a convergent subsequence. \label{th_210_b}
\end{enumerate}
\end{theorem}
\begin{proof}
~\begin{enumerate}[label=(\alph*)]
    \item Let $E$ be the range of $\{x_n\}$. If $E$ is finite then there is a $x \in E$ and a sequence $\{n_k\}$ with $n_1 < n_2 < n_3 < \cdots$, such that $x_{n_1} = x_{n_2} = \cdots = x$. The subsequence $\{x_{n_k}\}$ obtained converges to $x$. If $E$ is infinite, Theorem \ref{th_119} implies that $E$ has a limit point $x \in X$. Choose $n_1$ such that $d(x,x_{n_1}) < 1$, and let $n_2 > n_1$ be such that $d(x,x_{n_2}) < 1/2$. Continue this process and we have $\{x_{n_k}\}$ such that $d(x,x_{n_k}) < 1/k$, and clearly $x_{n_k} \to x$.
    
    \item This follows from \ref{th_210_a}, since by Theorem \ref{th_123}, every bounded subset of $\mathbb{R}^n$ belongs to a compact subset of $\mathbb{R}^n$.
\end{enumerate}
\end{proof}

\begin{remark}
If the bounded sequence belongs to $\mathbb{R}$ in \ref{th_210_b}, then the result is well known as Bolzano-Weierstrass theorem.
\end{remark}

\medskip

\begin{theorem}[Bolzano-Weierstrass]
Every bounded sequence of real numbers has a convergent subsequence.
\end{theorem}

\medskip

\begin{theorem}\label{th_212}
The subsequential limits of a sequence $\{x_n\}$ in a metric space $X$ forms a closed subset of $X$.
\end{theorem}
\begin{proof}
Let $E$ be the set of all subsequential limits of $\{x_n\}$ and suppose $x$ is a limit point in $E$. We need to show that $x \in E$.

Choose $n_1$ such that $x_{n_1} \neq X$. If no such point exists, then $E$ has only one point, hence closed. Let $\delta = d(x, x_{n_1})$. Suppose $n_1, \cdots, n_m$ are chosen. Since $x$ is a limit point in $E$, then there is a point $\widetilde{x} \in E$ such that $d(x, \widetilde{x}) < \delta/2^m$. Since $\widetilde{x} \in E$, there exists $n_{m+1} > n_m$ such that $d(\widetilde{x},x_{n_{m+1}}) < \delta/2^m$. Hence,
\begin{align*}
    d(x,x_{n_{m+1}}) \leq \frac{\delta}{2^{m-1}},
\end{align*}
for $m = 1,2,\cdots$, which implies $\{x_{n_k}\}$ converges to $x$. Hence $x \in E$.
\end{proof}


\medskip


\section{Cauchy Sequences}

\begin{definition}
A sequence $\{x_n\}$ in a metric space $X$ is said to be a Cauchy sequence if for every $\varepsilon > 0$, there is an integer $N > 0$ such that $d(x_n, x_m) < \varepsilon$ for all $n, m \geq N$.
\end{definition}

\medskip

\begin{definition}
Let $E$ be a nonempty subset of a metric space $X$, and let $S$ be the set of all real numbers of the the form $d(x,y)$ for $x,y \in E$. Then $\sup S$ is called the diameter of $E$, denoted by $\diam E$.
\end{definition}

\begin{remark}
If $\{x_n\}$ is a sequence in $X$ and if $E_n$ consists of the points $x_n, x_{n+1}, x_{n+2}, \cdots$, it is clear that $\{x_n\}$ is a Cauchy sequence if and only if $\lim_{n\to\infty} \diam E_n = 0$.
\end{remark}

\medskip

\begin{theorem}\label{th_213}
~\begin{enumerate}[label=(\alph*)]
    \item If $\overline{E}$ is the closure of a set $E \subset X$, then $\diam \overline{E} = \diam E$. \label{th_213_a}
    
    \item If $K_n$ is a sequence of compact sets in $X$ such that $K_n \supset K_{n+1}$ for $n = 1,2,3,\cdots$ and if
    \begin{align*}
        \lim_{n\to\infty} \diam K_n = 0,
    \end{align*}
    then $\bigcap^\infty_{n=1} K_n$ consists of exactly one point. \label{th_213_b}
\end{enumerate}
\end{theorem}
\begin{proof}
~\begin{enumerate}[label=(\alph*)]
    \item Since $E \subset \overline{E}$, then $\diam E \leq \diam \overline{E}$. For every $\varepsilon > 0$, let $x,y \in \overline{E}$, and there are $x',y' \in E$ such that $d(x,x') < \varepsilon$, $d(y,y') < \varepsilon$. Hence, 
    \begin{align*}
        d(x,y) \leq d(x,x') + d(x',y') + d(y',y) \leq d(x',y') + 2 \varepsilon,
    \end{align*}
    and it follows that $\diam \overline{E} \leq \diam E + 2 \varepsilon$. Since $\varepsilon$ is arbitrary, $\diam \overline{E} \leq \diam E$. Thus $\diam \overline{E} = \diam E$.
    
    \item Let $K = \bigcap^\infty_{n=1} K_n$ and by Theorem \ref{th_118}, $K$ is not empty. Suppose that $K$ contains more than one point, then $\diam K > 0$. However, for each $n$, $K_n \supset K$ so that $\diam K_n \geq \diam K$, which is a contradiction to $\lim_{n\to\infty} \diam K_n = 0$. 
\end{enumerate}
\end{proof}

\medskip

\begin{theorem}\label{th_214}
~\begin{enumerate}[label=(\alph*)]
    \item In any metric space $X$, every convergent sequence is a Cauchy sequence. \label{th_214_a}
    
    \item If $X$ is a compact metric space and if $\{x_n\}$ is a Cauchy sequence in $X$, then $\{x_n\}$ converges to some point of $X$. \label{th_214_b}
    
    \item In $\mathbb{R}^n$, every Cauchy sequence converges. \label{th_214_c}
\end{enumerate}
\end{theorem}
\begin{proof}
~\begin{enumerate}[label=(\alph*)]
    \item If $x_n \to x$ and for every $\varepsilon > 0$, there is an integer $N > 0$ such that $d(x,x_n) < \varepsilon$ for all $n \geq N$. Hence, 
    \begin{align*}
        d(x_n,x_m) \leq d(x_n,x) + d(x_m,x) < 2 \varepsilon,
    \end{align*}
    for $n,m \geq N$. Hence $\{x_n\}$ is a Cauchy sequence.
    
    \item Let $\{x_n\}$ be a Cauchy sequence in $X$. Let $E_n = \{x_n, x_{n+1}, \cdots\}$ for $n = 1,2,3,\cdots$. By Theorem \ref{th_213} \ref{th_213_b}, \begin{align}\label{th_214_equ1}
        \lim_{n\to\infty} \diam \overline{E}_n = 0.
    \end{align}
    Note that each $\overline{E}_n$ is compact by Theorem \ref{th_116} and also $\overline{E}_n \supset \overline{E}_{n+1}$. Also, Theorem \ref{th_213} \ref{th_213_b} implies there is a unique point $x \in X$ such that $x = \bigcap^\infty_{n=1} E_n$. 
    
    For every $\varepsilon > 0$, \eqref{th_214_equ1} implies that there is an integer $N > 0$ such that $\diam \overline{E}_n < \varepsilon$ for all $n \geq N$. Since $x \in \overline{E}_n$, it follows that $d(x,y) < \varepsilon$ for every $y \in \overline{E}_n$. Hence $d(x,x_n) < \varepsilon$ for all $n \geq N$, and this is precisely that $x_n \to x$. 
    
    \item Let $\{x_n\}$ be a Cauchy sequence in $\mathbb{R}^n$. Define $E_n$ as above. Then for $\varepsilon = 1$, there is an integer $N > 0$ such that $\diam \overline{E}_N < 1$. The range of $\{x_n\}$ is the union of $E_N$ and the finite set $\{x_1, \cdots, x_{N-1}\}$, and hence $\{x_n\}$ is bounded. Also, then range of $\{x_n\}$ is closed and hence compact, then \label{th_214_c} follows from \label{th_214_b}.
\end{enumerate}
\end{proof}

\medskip

\begin{definition}
A metric space $X$ in which every Cauchy sequence converges is said to be complete.
\end{definition}

\medskip

\begin{remark}
Being complete means every Cauchy sequence in $X$ also converges in $X$. Hence, Theorem \ref{th_214} implies that all compact metric and all Euclidean spaces (including $\mathbb{R}^n$) are complete. 
\end{remark}

\medskip

\begin{definition}
A sequence $\{x_n\}$ of real numbers is said to be
\begin{enumerate}[label=(\alph*)]
    \item monotonically increasing if $x_n \leq x_{n+1}$ for $n = 1,2,3,\cdots$;
    
    \item monotonically decreasing if $x_n \geq x_{n+1}$ for $n = 1,2,3,\cdots$.
\end{enumerate}
\end{definition}

\medskip

\begin{theorem}\label{th_215}
Suppose $\{x_n\}$ is monotonic. Then $\{x_n\}$ converges if and only if it is bounded.
\end{theorem}
\begin{proof}
Suppose $x_n \leq x_{n+1}$. Let $E$ be the range of $\{x_n\}$. If $\{x_n\}$ is bounded, let $s = \sup E$, then $s_n \leq s$ for all $n = 1,2,3,\cdots$. For every $\varepsilon > 0$, there is an integer $N > 0$ such that $s - \varepsilon < x_N < s$, for otherwise $s - \varepsilon$ would be the least upper bound. Since $\{x_n\}$ increases, for all $n \geq N$, $s - \varepsilon < x_n < s$, which shows that $\{x_n\}$ converges.

The converse follows from Theorem \ref{th_21} \ref{th_21_c}.
\end{proof}


\medskip



\section{Upper and Lower Limits}

\begin{definition}\label{def_28}
Let $\{x_n\}$ be a sequence of real numbers. Let $E$ be the set of all $x \in \overline{\mathbb{R}}$ such that $x_{n_k} \to x$ for some subsequence $\{x_{n_k}\}$. The set contains all subsequential limits, plus possibly the numbers $+\infty$, $-\infty$. Let
\begin{align*}
    x^* = \sup E, \qquad x_* = \inf E.
\end{align*}
The numbers $x^*$ and $x_*$ are called the upper and lower limits of $\{x_n\}$, we use the notation
\begin{align*}
    \limsup_{n\to\infty} x_n = x^*, \qquad \liminf_{n\to\infty} x_n = x_*.
\end{align*}
\end{definition}

\begin{remark}
The upper limit $\limsup_{n\to\infty} x_n$ is the largest possible limit (including $+\infty$) of any subsequence of $\{x_n\}$, and the lower limit $\liminf_{n\to\infty} x_n$ is the smallest possible limit (including $-\infty$). Hence we could also define $\limsup$ and $\liminf$ as below:
\begin{enumerate}[label=(\alph*)]
    \item If $\{x_n\}$ is bounded from above, then \begin{align*}
        \limsup_{n\to\infty} x_n \coloneqq \inf_{n \in \mathbb{N}} \sup_{k \geq n} x_k,
    \end{align*}
    and this number exists since $\{x_n\}$ is bounded from above. If $\{x_n\}$ is not bounded above, then $\limsup x_n \coloneqq +\infty$.
    
    \item If $\{x_n\}$ is bounded from below, then \begin{align*}
        \liminf_{n\to\infty} x_n \coloneqq \sup_{n \in \mathbb{N}} \inf_{k \geq n} x_k.
    \end{align*}
    If $\{x_n\}$ is not bounded from below, then $\liminf x_n \coloneqq -\infty$.
\end{enumerate}
\end{remark}

\medskip

\begin{lemma}
Let $\{x_n\} \subset \mathbb{R}$ be a sequence.
\begin{enumerate}[label=(\alph*)]
    \item If $\sup_{k \geq n} x_k$ exists, then
    \begin{align*}
        \limsup_{n\to\infty} x_n = \lim_{n\to\infty} \sup_{k \geq n} x_k.
    \end{align*}
    
    \item If $\inf_{k \geq n} x_k$ exists, then
    \begin{align*}
        \liminf_{n\to\infty} x_n = \lim_{n\to\infty} \inf_{k \geq n} x_k.
    \end{align*}
\end{enumerate}
\end{lemma}
\begin{proof}
~\begin{enumerate}[label=(\alph*)]
    \item Let $a_n = \sup_{k\geq n}x_k$. If $\{a_n\}$ is not bounded from above, then $\{x_n\}$ is not bounded above, hence $\lim a_n = \limsup x_n = + \infty$. 
    
    If $\{a_n\}$ is bounded from above, then it is a bounded and monotonically decreasing sequence. By Theorem \ref{th_215}, $\{a_n\}$ is convergent and
    \begin{align*}
        \lim_{n\to\infty} a_n = \inf_{n \in \mathbb{N}} \sup_{k \geq n} x_k = \limsup_{n\to\infty} x_n.
    \end{align*}
    
    \item By the similar argument as above.
\end{enumerate}
\end{proof}

\medskip

\begin{theorem}\label{th_216}
Let $\{x_n\} \subset \mathbb{R}$ be a sequence. Let $E$ and $x^*$ be the same as in Definition \ref{def_28}. Then $X^*$ has the following properties:
\begin{enumerate}[label=(\alph*)]
    \item $x^* \in E$. \label{th_216_a}
    
    \item If $x > x^*$, there is an integer $N$ such that $x_n < x$ for $n \geq N$. \label{th_216_b}
\end{enumerate}
Moreover, $x^*$ is the only number which satisfies \ref{th_216_a} and \ref{th_216_b}. An analogous result is true for $x_*$.
\end{theorem}
\begin{proof}
~\begin{enumerate}[label=(\alph*)]
    \item If $x^* = + \infty$, then $E$ is not bounded and hence $\{x_n\}$ is not bounded and there is a subsequence $\{x_{n_k}\}$ such that $x_{n_k} \to + \infty$. If $x^*$ is finite, then $E$ is bounded above, and hence at least one subsequential limit exists. \ref{th_216_a} follows from Theorem \ref{th_212} and \ref{th_113}. If $x^* = - \infty$, then $E$ only contains one element $- \infty$, and there is no subsequential limit. Hence for any $M$, $x_n > M$ for at most a finite number of values of $n$, so $x_n \to - \infty$.
    
    \item Suppose there is a number $x > x^*$ such that $x_n \geq x$ for infinitely many $n$, then there is $y \in E$ such that $y \geq x > x^*$, contradiction.
\end{enumerate}
To show the uniqueness, suppose there are two numbers $x_1$ and $x_2$ which satisfy \ref{th_216_a} and \ref{th_216_b}, and suppose $x_1 < x_2$. Choose $x$ such that $x_1 < x < x_2$. Since $x_1$ satisfies \ref{th_216_b}, we have $x_n < x$ for all $n \geq N$, hence $x_2$ cannot satisfy \ref{th_216_a}.
\end{proof}

\medskip

We will also present another proof of part \ref{th_216_a} when $\{x_n\}$ is bounded in the above theorem.

\medskip

\begin{proof}[Second Proof of \ref{th_216_a}]
Let $x^* = \sup E$. It suffices to show that there is a subsequence converging to $x^*$. 

Let $k \in \mathbb{N}$. Since $x - 1/k < x^*$, there is a subsequence with limit larger than $x^* - 1/k$, in particular, infinitely many $x_n$ are larger than $x^* - 1/k$. Hence there is $n_1$ such that $x_{n_1} > x^* - 1$, and there is $n_2 > n_1$ such that $x_{n_2} > x^* - 1/2$. Continue this process and we can have a subsequence $\{x_{n_k}\}$ such that $x_{n_k} > x^* - 1/k$ for $k = 1,2,3,\cdots$. Also, $\{x_{n_k}\}$ is bounded, os it has a convergent subsequence $\{x_{n_{k_l}}\}$. Clearly, $\lim_{l\to\infty} x_{n_{k_l}} \leq x^*$. 

Note that $k_l \geq l$, so for $l \geq i$
\begin{align*}
    x_{n_{k_l}} \geq x^* - \frac{1}{k_l} \geq x^* - \frac{1}{l} \geq x^* - \frac{1}{i},
\end{align*}
and hence
\begin{align*}
    x^* \geq \lim_{l\to\infty} x_{n_{k_l}} \geq x^* - \frac{1}{i},
\end{align*}
letting $i \to \infty$ implies $\lim_{l\to\infty} x_{n_{k_l}} = x^*$.
\end{proof}



\medskip


\section{Series}

\begin{definition}
Given a sequence $\{x_n\}$, we use the notation
\begin{align*}
    \sum^m_{n=k} x_n
\end{align*}
to denote the sum $x_k + a_{k+1} + \cdots + a_m$. With $\{x_n\}$ we associate a sequence $\{s_n\}$, which is defined as
\begin{align*}
    s_n = \sum^n_{k=1} x_k.
\end{align*}
For $\{s_n\}$ we also use the symbolic expression 
\begin{align}\label{sec_25_equ1}
    \sum^\infty_{n=1} x_n
\end{align}
to denote $x_1 + x_2 + x_3 + \cdots$.
\end{definition}

The symbol \eqref{sec_25_equ1} is called an {\em infinite series}, or just a {\em series}. $s_n$ are called the {\em partial sums} of the series. If $\{s_n\}$ converges to $x$, then we say the series converges, and write
\begin{align*}
    \sum^\infty_{n=1} x_n = x.
\end{align*}

\medskip

\begin{theorem}
Given a sequence $\{x_n\}$, $\sum x_n$ converges if and only if for every $\varepsilon > 0$ there is an integer $N > 0$ such that if $m \geq n \geq N$,
\begin{align*}
    \left|\sum^m_{k=n} x_k\right| \leq \varepsilon.
\end{align*}
This is called the Cauchy criterion. In particular, letting $m = n$ implies that for $n \geq N$, $\left|x_n\right| \leq \varepsilon$. In other words:
\end{theorem}

\medskip

\begin{theorem}\label{th_218}
If $\sum x_n$ converges, then $\lim_{n\to\infty} x_n = 0$.
\end{theorem}

\medskip

\begin{remark}
In general, the condition $\lim_{n\to\infty} x_n = 0$ is not sufficient to ensure the convergence of $\sum x_n$. For example, the series
\begin{align*}
    \sum^\infty_{n=1} \frac{1}{n}
\end{align*}
diverges, which we will prove it later. 
\end{remark}

\medskip

\begin{theorem}\label{th_219}
If $x_n \geq 0, n = 1,2,3,\cdots$, then $\sum x_n$ converges if and only if the sequence of partial sums is bounded. 
\end{theorem}
\begin{proof}
$x_n \geq 0$ implies that $s_n = \sum^n_{k=1} x_k$ is an increasing sequence, and hence $\{s_n\}$ converges if and only if it is bounded.
\end{proof}

\medskip

\begin{theorem}[Comparison Test]\label{th_220}
~\begin{enumerate}[label=(\alph*)]
    \item If $\left|x_n\right| \leq c_n$ for all $n > N$, where $N$ is some fixed integer, and if $\sum c_n$ converges, then $\sum x_n$ converges. \label{th_220_a}
    
    \item If $x_n \geq d_n \geq 0$ for all $n \geq N$, and if $\sum d_n$ diverges, then $\sum x_n$ diverges. \label{th_220_b}
\end{enumerate}
\end{theorem}
\begin{proof}
Part \ref{th_220_b} is obvious, and we only prove part \ref{th_220_a}. Given $\varepsilon > 0$, there is an integer $\widetilde{N} \geq N$ such that for all $m \geq n \geq \widetilde{N}$, 
\begin{align*}
    \sum^m_{k=n} c_k < \varepsilon,
\end{align*}
hence
\begin{align*}
    \left|\sum^m_{k=n} x_k\right| \leq \sum^m_{k=n} \left|x_k\right| \leq \sum^m_{k=n} c_k < \varepsilon,
\end{align*}
which implies that $\{x_n\}$ converges.
\end{proof}

\medskip



\section{Series of Nonnegative Terms}

\begin{theorem}[Geometric series]\label{th_221}
If $0 \leq x < 1$, then
\begin{align*}
    \sum^\infty_{n=0} x^n = \frac{1}{1 - x}.
\end{align*}
If $x > 1$, the series diverges.
\end{theorem}
\begin{proof}
If $0 \leq x < 1$, we have
\begin{align*}
    s_n \coloneqq \sum^n_{k=1} x^k = \frac{1 - x^{n+1}}{1 - x},
\end{align*}
and hence letting $n \to \infty$ implies the result. For $x = 1$, we have $\sum^\infty_{n=0} x^n = 1 + 1 + 1 + \cdots$, which diverges. 
\end{proof}

\medskip

\begin{theorem}[Cauchy Condensation Test]\label{th_222}
Suppose $a_1 \geq a_2 \geq a_3 \geq \cdots \geq 0$. Then the series $\sum^\infty_{n=0} a_n$ converges if and only if the series $\sum^\infty_{n=0} 2^n a_{2^n}$ converges.
\end{theorem}
\begin{proof}
By Theorem \ref{th_219}, it suffices to show the boundness of the partial sums. Let
\begin{align*}
    s_n & = a_1 + a_2 + \cdots + a_n, \\
    t_k & = a_1 + 2a_2 + \cdots + 2^k a_{2^k}.
\end{align*}
For $n < 2^k$, we have
\begin{align}\label{th_222_equ1}
    s_n & \leq a_1 + (a_2 + a_3) + \cdots + (a_{2^k} + \cdots + a_{2^{k+1}-1}) \\
    & \leq a_1 + 2a_2 + \cdots + 2^k a_{2^k} = t_k,
\end{align}
and hence $s_n \leq t_k$. 

On the other hand, if $n > 2^k$,
\begin{align}\label{th_222_equ2}
    s_n & \geq a_1 + a_2 + (a_3 + a_4) + \cdots + (a_{2^{k-1}+1} + \cdots a_{2^k}) \\
    & \geq \frac{1}{2}a_1 + a_2 + 2 a_4 + \cdots + 2^{k-1} a_{2^k} = \frac{1}{2} t_k.
\end{align}
Hence \eqref{th_222_equ1} and \eqref{th_222_equ2} implies that $\{s_n\}$ and $\{t_k\}$ are either both bounded or both unbounded. This completes the proof.
\end{proof}

\medskip

\begin{theorem}[$p$-Series Test]\label{th_223}
$\displaystyle \sum \frac{1}{n^p}$ converges if $p > 1$ and diverges if $p \leq 1$.
\end{theorem}
\begin{proof}
If $p \leq 1$, then $1/n^p \geq 1/n$, and since $\sum 1/n$ diverges, $\sum 1/n^p$ also diverges. For $p > 1$, we apply the Cauchy Condensation Test, and let $a_n = 1/n^p$, then
\begin{align*}
    \sum^\infty_{k=0} 2^k \frac{1}{2^{kp}} = \sum^\infty_{k=0} 2^{(1-p)k}.
\end{align*}
Now $2^{1-p} < 1$ if and only if $1 - p < 0$, and the result follows by comparison with the geometric series (take $x = 2^{1-p}$ in Theorem \ref{th_221}).
\end{proof}

\medskip

\begin{theorem}
If $p > 1$, the series
\begin{align*}
    \sum^\infty_{n=2} \frac{1}{n \left(\log n\right)^p} 
\end{align*}
converges and if $p \leq 1$, the series diverges.
\end{theorem}
\begin{proof}
Let $a_n = 1/\left(n \left(\log n\right)^p\right)$, then
\begin{align}\label{th_224_equ1}
    \sum^\infty_{n=2} 2^n a_{2^n} = \left(\frac{1}{\log 2}\right)^p \sum^\infty_{n=2} \frac{1}{n^p},
\end{align}
and by Theorem \ref{th_223}, \eqref{th_224_equ1} converges if and only if $p > 1$ and then the theorem follows from the Cauchy Condensation Test in Theorem \ref{th_222}.
\end{proof}

\medskip



\section{Natural Number $e$ and Natural Logarithm}

\begin{theorem}
The sequence $a_n = \left(1 + 1/n\right)^n$ is strictly increasing and the sequence $b_n = \left(1 + 1/n\right)^{n+1}$ is strictly decreasing, and both sequences converges to the same limit. We denote this limit by
\begin{align*}
    e = \lim_{n\to\infty} \left(1 + \frac{1}{n}\right)^n = \lim_{n\to\infty} \left(1 + \frac{1}{n}\right)^{n+1}.
\end{align*}
\end{theorem}
\begin{proof}
First we prove that $a_n$ is strictly increasing, and by the Bernoulli formula, we have
\begin{align*}
    \frac{a_{n+1}}{a_n} & = \frac{\left(\frac{n + 2}{n + 1}\right)^{n + 1}}{\left(\frac{n + 1}{n}\right)^{n}}  = \left(\frac{n^2 + 2n}{n^2 + 2n + 1}\right)^n \frac{n + 2}{n + 1} \\
    & = \left(1 - \frac{1}{n^2 + 2n + 1}\right)^n \frac{n + 2}{n + 1} \\
    & \geq \left(1 - \frac{n}{n^2 + 2n + 1}\right) \frac{n + 2}{n + 1} \\
    & = \frac{n^3 + 3n^2 + 3n + 2}{n^3 + 3n^2 + 3n + 1} > 1.
\end{align*}
And similarly, we can show $b_n$ is decreasing. Since $a_n \leq b_n$ for all $n$, we have
\begin{align*}
    2 = a_1 < a_2 < \cdots < a_n < \cdots < b_n < \cdots < b_2 < b_1 = 4.
\end{align*}
Hence $\{a_n\}$ is increasing and bounded from above, so convergent, and also similar for $\{b_n\}$. Thus, 
\begin{align*}
    \frac{\lim_{n\to\infty} b_n}{\lim_{n\to\infty} a_n} = \lim_{n\to\infty} \frac{b_n}{a_n} = \lim_{n\to\infty} 1 + \frac{1}{n} = 1,
\end{align*}
which implies
\begin{align*}
    \lim_{n\to\infty} a_n = \lim_{n\to\infty} b_n.
\end{align*}
\end{proof}

\begin{remark}
Since $e$ is the common limit of $\{a_n\}$ and $\{b_n\}$, we have
\begin{align}\label{remark_27_equ1}
    \left(1 + \frac{1}{n}\right)^n < e < \left(1 + \frac{1}{n}\right)^{n+1}.
\end{align}
\end{remark}

\medskip

\begin{theorem}
$\displaystyle e = \sum^\infty_{n=0} \frac{1}{n!}$.
\end{theorem}
\begin{proof}
With $0! = 1$, let
\begin{align*}
    x_n = \left(1 + \frac{1}{n}\right)^n, \quad y_n = \sum^n_{k=0} \frac{1}{k!}.
\end{align*}
And the binomial formula yields
\begin{align*}
    x_n & = 1^n + \binom{n}{1} \frac{1}{n} + \binom{n}{2} \frac{1}{n^2} + \cdots + \binom{n}{n-1} \frac{1}{n^{n-1}} + \frac{1}{n^n} \\
    & = 1 + 1 + \frac{n(n-1)}{2!} \frac{1}{n^2} + \cdots + \frac{n!}{k!(n-k)!} \frac{1}{n^k} + \cdots + \frac{1}{n^n} \\
    & = 1 + 1 + \frac{n-1}{n} \frac{1}{2!} + \frac{(n-1)(n-2)}{n^2} \frac{1}{3!} + \cdots + \frac{(n-1)!}{n^{n-1}} \frac{1}{n!} \leq y_n.
\end{align*}
On the other hand, for $n \geq k$ we have
\begin{align*}
    x_n \geq 1 + 1 + \frac{n-1}{n} \frac{1}{2!} + \cdots + \frac{(n-1)(n-2)\cdots(n-k+1)}{n^k} \frac{1}{k!},
\end{align*}
with this fixed $k$, letting $n\to\infty$ on both sides yields
\begin{align*}
    e = \lim_{n\to\infty} x_n \geq 1 + 1 + \frac{1}{2!} + \frac{1}{3!} + \cdots + \frac{1}{k!} = y_k.
\end{align*}
Hence $e \leftarrow x_n \leq y_n \leq e$, which implies
\begin{align*}
    \sum^\infty_{n=0} \frac{1}{n!} = \lim_{n\to\infty} y_n = e.
\end{align*}
\end{proof}

\medskip

\begin{definition}
The natural logarithm is defined by
\begin{align*}
    \ln x = \log x = \log_e x.
\end{align*}
\end{definition}

\medskip

\begin{lemma}
$\displaystyle \frac{1}{n + 1} < \ln \left(1 + \frac{1}{n}\right) < \frac{1}{n}$ for $n = 1,2,3,\cdots$.
\end{lemma}
\begin{proof}
Taking $\ln$ function on both sides of the inequality \eqref{remark_27_equ1} implies 
\begin{align*}
    n \ln\left(1 + \frac{1}{n}\right) < 1 < (n+1) \ln \left(1 + \frac{1}{n}\right),
\end{align*}
and the lemma follows.
\end{proof}

\medskip



\section{The Root and Ratio Test}

\begin{definition}
We say a series $\sum^\infty_{n=1} a_n$ is absolutely convergent if $\sum^\infty_{n=1} \left|a_n\right|$ converges.
\end{definition}

\medskip

\begin{theorem}
If $\sum^\infty_{n=1} a_n$ is absolutely convergent, then it is convergent.
\end{theorem}
\begin{proof}
Clearly $\left|a_n\right| \leq \left|a_n\right|$ and by the Comparison Test (Theorem \ref{th_220}), the theorem follows.
\end{proof}

\medskip

Now we talk about two useful theorems that can be used to determine if the series converges, {\em Root Test} (or {\em Cauchy Test}) and {\em Ratio Test} (or {\em d'Alembert Test}).

\medskip

\begin{theorem}[Root Test]
Given $\sum^\infty_{n=1} a_n$.
\begin{enumerate}[label=(\alph*)]
    \item If $\limsup_{n\to\infty}\displaystyle  \sqrt[n]{\left|a_n\right|} < 1$, then $\sum^\infty_{n=1} a_n$ converges absolutely.
    
    \item If $\limsup_{n\to\infty}\displaystyle  \sqrt[n]{\left|a_n\right|} > 1$, then $\sum^\infty_{n=1} a_n$ diverges.
\end{enumerate}
\end{theorem}
\begin{remark}
If $\limsup_{n\to\infty}\displaystyle  \sqrt[n]{\left|a_n\right|} = 1$, the test gives no information. For example, consider the series
\begin{align}\label{remark_28_equ1}
    \sum^\infty_{n=1} \frac{1}{n}, \quad \sum^\infty_{n=1} \frac{1}{n^2}.
\end{align}
For each of these sequences $ \limsup_{n\to\infty} \displaystyle\sqrt[n]{\left|a_n\right|} = 1$, but the first diverges and the second converges.
\end{remark}
\begin{proof}
Let $\alpha = \limsup_{n\to\infty}\displaystyle  \sqrt[n]{\left|a_n\right|}$. If $\alpha < 1$, then there is $\beta$ such that $\alpha < \beta < 1$. Then by Theorem \ref{th_216} \ref{th_216_b}, there is an integer $N > 0$ such that for all $n \geq N$, 
\begin{align*}
    \sqrt[n]{\left|a_n\right|} < \beta,
\end{align*}
which is equivalent to that $\left|a_n\right| < \beta^n$ for $n \geq N$. By Theorem \ref{th_221}, $\sum \beta^n$ converges. And the absolute convergence of $\sum a_n$ follows from the Comparison Test. 

If $\alpha > 1$, then again by Theorem \ref{th_216}, there is a subsequence $\{a_{n_k}\}$ such that 
\begin{align*}
    \sqrt[n_k]{a_{n_k}} \to \alpha,
\end{align*}
and hence $\left|a_n\right| > 1$ for infinitely many numbers of $n$, so the condition $a_n \to 0$, necessary for convergence of $\sum a_n$, does not hold (Theorem \ref{th_218}).
\end{proof}

\medskip

\begin{theorem}[Ratio Test]
Given $\sum^\infty_{n=1} a_n$.
\begin{enumerate}[label=(\alph*)]
    \item If $\lim_{n\to\infty}\displaystyle  \left|\frac{a_{n+1}}{a_n}\right| < 1$, then $\sum^\infty_{n=1} a_n$ converges absolutely.
    
    \item If $\lim_{n\to\infty}\displaystyle  \left|\frac{a_{n+1}}{a_n}\right| > 1$, then $\sum^\infty_{n=1} a_n$ diverges.
\end{enumerate}
\end{theorem}

\begin{remark}
If $\lim_{n\to\infty}\displaystyle  \left|\frac{a_{n+1}}{a_n}\right| = 1$, the test gives no information, and the series in \eqref{remark_28_equ1}  this.
\end{remark}

\begin{proof}
Let $\alpha = \lim_{n\to\infty} \left|a_{n+1}/a_n\right|$. If $\alpha < 1$, then for $0 < \varepsilon < 1 - a$, there is an integer $N > 0$ such that for all $n \geq N$, 
\begin{align*}
    \left|\frac{a_{n+1}}{a_n}\right| < a + \varepsilon,
\end{align*}
then there is $\beta$ such that $a + \varepsilon < \beta < 1$, and hence for all $n \geq N$,
\begin{align*}
    \left|\frac{a_{n+1}}{a_n}\right| < \beta.
\end{align*}
In particular, $\left|a_{n}\right| < \left|a_N\right| \beta^{-N} \cdot \beta^n$ for $n \geq N$. Since $\sum \beta^n$ converges,  and $\sum a_n$ also converges absolutely by the Comparison Test.

If $\alpha > 1$, then there is an integer $N > 0$ such that for all $n \geq N$, $\left|a_{n+1}\right| > \left|a_n\right|$ and hence the condition $a_n \to 0$ does not hold. 
\end{proof}

\medskip

\begin{theorem}
For any sequence $\{a_n\}$ of positive numbers,
\begin{align*}
    \liminf_{n\to\infty} \frac{a_{n+1}}{a_n} & \leq \liminf_{n\to\infty} \sqrt[n]{a_n}, \\
    \limsup_{n\to\infty} \sqrt[n]{a_n} & \leq \limsup_{n\to\infty} \frac{a_{n+1}}{a_n}.
\end{align*}
\end{theorem}
\begin{proof}
For the second inequality, let $\alpha = \limsup_{n\to\infty} a_{n+1}/a_n$. If $\alpha = + \infty$, we are done. If $\alpha < \infty$, let $\beta > \alpha$. There is an integer $N > 0$ such that for all $n \geq N$,
\begin{align*}
    \frac{a_{n+1}}{a_n} \leq \beta.
\end{align*}
For $n \geq N$, we have $a_n \leq a_N \beta^{-N} \beta^n$, and hence $\sqrt[n]{a_n} \leq \sqrt[n]{a_N \beta^{-N}} \beta$. Since $a_N \beta^{-N} > 0$, we have
\begin{align*}
    \limsup_{n\to\infty} \sqrt[n]{a_n} \leq \limsup_{n\to\infty} \sqrt[n]{a_N \beta^{-N}} \beta = \beta,
\end{align*}
and this holds for all $\beta \geq \alpha$, thus 
\begin{align*}
    \limsup_{n\to\infty} \sqrt[n]{a_n} \leq \alpha.
\end{align*}

The proof of the first inequality is similar.
\end{proof}

\begin{remark}
This theorem shows that the Ratio Test is easier to apply than the Root Test, since it is usually easier to compute ratio than $n$th roots. However, the Ratio Test has wider application, i.e., when the Ratio Test shows convergence, the Root Test does too, and when the Root Test is inconclusive, the Ratio Test is too. We show some examples below to see this explicitly. 
\end{remark}

\medskip

\begin{example}
~\begin{enumerate}[label=(\alph*)]
    \item Consider the series
    \begin{align*}
        \sum^\infty_{n=1} a_n = \frac{1}{2} + \frac{1}{2} + \frac{1}{2^2} + \frac{1}{3^2} + \frac{1}{2^3} + \frac{1}{3^3} + \frac{1}{2^4} + \frac{1}{3^4} + \cdots,
    \end{align*}
    and we have
    \begin{align*}
        \liminf_{n\to\infty} \frac{a_{n+1}}{a_n} = \lim_{n\to\infty} \frac{2^n}{3^n} = 0, \quad & \liminf_{n\to\infty} \sqrt[n]{a_n} = \lim_{n\to\infty} \sqrt[2n]{\frac{1}{3^n}} = \frac{1}{\sqrt{3}}, \\
        \limsup_{n\to\infty} \frac{a_{n+1}}{a_n} = \lim_{n\to\infty} \frac{3^n}{2^{n+1}} = + \infty, \quad & \limsup_{n\to\infty} \sqrt[n]{a_n} = \lim_{n\to\infty} \sqrt[2n]{\frac{1}{2^n}} = \frac{1}{\sqrt{2}}.
    \end{align*}
    The Root Test shoes convergence, however, the Ratio Test does not apply here.
    
    \item The same it true for the series
    \begin{align*}
        \sum^\infty_{n=1} a_n = \frac{1}{2} + 1 + \frac{1}{8} + \frac{1}{4} + \frac{1}{32} + \frac{1}{16} + \frac{1}{128} + \frac{1}{64} + \cdots,
    \end{align*}
    and we have
    \begin{align*}
        \liminf_{n\to\infty} \frac{a_{n+1}}{a_n} = \frac{1}{8}, \quad \limsup_{n\to\infty} \frac{a_{n+1}}{a_n} = 2,
    \end{align*}
    but we also have
    \begin{align*}
        \limsup_{n\to\infty} \sqrt[n]{a_n} = \lim_{n\to\infty} \sqrt[n]{\frac{1}{2^n}} = \frac{1}{2}, \quad \liminf_{n\to\infty} \sqrt[n]{a_n} = \lim_{n\to\infty} \sqrt[n+2]{\frac{1}{2^n}} = \frac{1}{2}.
    \end{align*}
\end{enumerate}
\end{example}

\medskip

\begin{theorem}
Let $\{a_n\}, \{b_n\}$ be two sequences such that $a_n, b_n > 0$, and there is an integer $N > 0$ such that for all $n \geq N$,
\begin{align*}
    \frac{a_{n+1}}{a_n} \leq  \frac{b_{n+1}}{b_n}.
\end{align*}
If $\sum^\infty_{n=1} b_n$ converges, then $\sum^\infty_{n=1} a_n$ converges.
\end{theorem}
\begin{proof}
Let $c_n = a_n/b_n$, then for all $n \geq N$,
\begin{align*}
    c_{n+1} = \frac{a_{n+1}}{b_{n+1}} \leq \frac{a_n}{b_n} = c_n,
\end{align*}
which implies $\{c_n\}$ is decreasing starting from $n = N$. Hence $\{c_n\}$ is bounded, that is there is $M$ such that $c_n \leq M$ for all $n$. Hence $a_n = c_n b_n \leq M b_n$. Since $\sum^\infty_{n=1} Mb_n = M \sum^\infty_{n=1} b_n$ converges, by the Comparison Test, $\sum^\infty_{n=1} a_n$ converges too.
\end{proof}

\medskip


\section{Power Series}

\begin{definition}
Given a sequence $\{c_n\}$ of complex numbers, the series
\begin{align*}
    \sum^\infty_{n=0} c_n z^n
\end{align*}
is called a power series. The numbers $c_n$ are called the coefficients of the series, $z$ is a complex number.
\end{definition}

\medskip

\begin{theorem}[Cauchy-Hadamard]
Given $\sum^\infty_{n=0} c_n z^n$, let
\begin{align*}
    \alpha = \limsup_{n\to\infty} \sqrt[n]{\left|c_n\right|}, \quad R = \frac{1}{\alpha}.
\end{align*}
If $\alpha = 0, R = + \infty$, if $\alpha = + \infty, R = 0$. Then $\sum^\infty_{n=0} c_n z^n$ converges if $\left|z\right| < R$ and diverges if $\left|z\right| > R$.
\end{theorem}
\begin{proof}
Let $a_n = c_n z^n$, and applying the Root Test yields
\begin{align*}
    \limsup_{n\to\infty} \sqrt[n]{\left|a_n\right|} = \left|z\right| \cdot \limsup_{n\to\infty} \sqrt[n]{\left|c_n\right|} = \frac{\left|z\right|}{R}.
\end{align*}
Hence $\left|z\right|/R < 1$ implies the convergence and $\left|z\right|/R > 1$ implies the divergence.
\end{proof}

\begin{remark}
$R$ is called the radius of convergence of $\sum^\infty_{n=0} c_n z^n$.
\end{remark}

\medskip


\section{Summation by Parts}

\begin{theorem}
Given two sequences $\{a_n\}, \{b_n\}$, let
\begin{align*}
    A_n = \sum^n_{k=0} a_k,
\end{align*}
if $n \geq 0$. Also let $A_{-1} = 0$. Then, if $0 \leq m \leq n$, we have
\begin{align}\label{th_233_equ1}
    \sum^n_{k=m} a_k b_k = \sum^{n-1}_{k=m} A_k(b_k - b_{k+1}) + A_n b_n - A_{m-1} b_m.
\end{align}
\end{theorem}
\begin{proof}
\begin{align*}
    \sum^n_{k=m} a_k b_k = \sum^n_{k=m} (A_k - A_{k-1}) b_k = \sum^n_{k=m} A_k b_k - \sum^{n-1}_{k=m-1} A_k b_{k+1},
\end{align*}
and the last term is clearly equal to the right hand side of \eqref{th_233_equ1}.
\end{proof}

\medskip

\begin{theorem}\label{th_234}
Suppose
\begin{enumerate}[label=(\alph*)]
    \item the partial sum $A_n$ of $\sum^\infty_{n=0} a_n$ form a bounded sequence;
    
    \item $b_0 \geq b_1 \geq b_2 \geq \cdots$;
    
    \item $\lim_{n\to\infty} b_n = 0$.
\end{enumerate}
Then $\sum^\infty_{n=0} a_n b_n$ converges.
\end{theorem}
\begin{proof}
Let $M > 0$ be such that $\left|A_n\right| \leq M$ for all $n$. Since $\lim_{n\to\infty} b_n = 0$ and $\{b_n\}$ is decreasing, for every $\varepsilon > 0$, there is an integer $N > 0$ such that $b_N \leq \varepsilon/(2M)$. Now for $n \geq m \geq N$, we have
\begin{align*}
    \left|\sum^n_{k=m} a_nb_n\right| & = \left|\sum^{n-1}_{k=m} A_k(b_k - b_{k+1}) + A_n b_n - A_{m-1} b_m\right| \\
    & \leq M \left|\sum^{n-1}_{k=m} (b_k - b_{k+1}) + b_n - b_m\right| \\
    & = 2M b_m \leq 2M b_N \leq \varepsilon.
\end{align*}
Now convergence of $\sum^\infty_{n=0} a_n b_n$ follows from the Cauchy criterion.
\end{proof}

\medskip

\begin{theorem}[Leibnitz]
Suppose
\begin{enumerate}[label=(\alph*)]
    \item $\left|c_1\right| \geq \left|c_2\right| \geq \left|c_2\right| \geq \cdots$;
    
    \item $c_{2k-1} \geq 0, c_{2k} \leq 0, k = 1,2,3,\cdots$;
    
    \item $\lim_{n\to\infty} c_n = 0$.
\end{enumerate}
Then $\sum^\infty_{n=0} c_n$ converges.
\end{theorem}
\begin{proof}
Apply Theorem \ref{th_234} with $a_n = (-1)^{n+1}$ and $b_n = \left|c_n\right|$.
\end{proof}

\medskip

\begin{theorem}
Suppose the radius of convergence of $\sum^\infty_{n=0} c_n z^n$ is $1$, and suppose $c_0 \geq c_1 \geq c_2 \geq \cdots$, $\lim_{n\to\infty} c_n = 0$. Then $\sum^\infty_{n=0} c_n z^n$ converges at every point on the circle $\left|z\right| = 1$, expect possibly at $z = 1$.
\end{theorem}
\begin{proof}
Let $a_n = z^n$, $b_n = c_n$, and if $\left|z\right| = 1, z \neq 1$, we have
\begin{align*}
    \left|A_n\right| = \left|\sum^n_{k=0} z^k\right| = \left|\frac{1 - z^{n+1}}{1 - z}\right| \leq \frac{2}{\left|1 - z\right|}.
\end{align*}
Hence the assumptions of Theorem \ref{th_234} are satisfied and the convergence follows.
\end{proof}

\medskip



\section{Addition and Multiplication of Series}

\begin{theorem}
If $\sum^\infty_{n=0} a_n = A$, and $\sum^\infty_{n=0} b_n = B$, then $\sum^\infty_{n=0} (a_n + b_n) = A + B$, and $\sum^\infty_{n=0} ca_n = cA$, for any fixed $c$.
\end{theorem}
\begin{proof}
Let
\begin{align*}
    A_n = \sum^n_{k=0} a_k, \quad B_n = \sum^n_{k=0} b_k.
\end{align*}
Then
\begin{align*}
    A_n + B_n = \sum^n_{k=0} (a_k + b_k).
\end{align*}
Since $\lim_{n\to\infty} A_n = A$ and $\lim_{n\to\infty} B_n = B$, we have 
\begin{align*}
    \lim_{n\to\infty}(A_n + B_n) = A + B.
\end{align*}
Proof of the second equality is easy.
\end{proof}

\medskip

\begin{theorem}
Given $\sum^\infty_{n=0} a_n$ and $\sum^\infty_{n=10} b_n$, let
\begin{align*}
    c_n = \sum^n_{k=0} a_k b_{n-k}, \,\, n = 1,2,3,\cdots,
\end{align*}
and call $\sum^\infty_{n=0} c_n$ the product of the two given series.
\end{theorem}


\begin{theorem}
If the series $\sum^\infty_{n=0} a_n$ converges absolutely and the series $\sum^\infty_{n=0} b_n$ converges, then
\begin{align*}
    \left(\sum^\infty_{n=0} a_n\right) \left(\sum^\infty_{n=0} b_n\right) = \sum^\infty_{n=0} c_n,
\end{align*}
where 
\begin{align*}
    c_n = \sum^n_{k=0} a_k b_{n-k}.
\end{align*}
\end{theorem}
\begin{proof}
Let
\begin{align*}
    A_n = \sum^n_{k=0} a_n, \quad B_n = \sum^n_{k=0} b_n, \quad C_n = \sum^n_{k=0} c_n, 
\end{align*}
and $A = \sum^\infty_{n=0} a_n, B = \sum^\infty_{n=0} b_n$. Also, we define $\beta_n = B_n - B$. Then,
\begin{align*}
    C_n & = a_0b_0 + (a_0b_1 + a_1b_0) + \cdots + (a_0b_n + a_1b_{n-1} + \cdots + a_nb_0) \\
    & = a_0 B_n + a_1 B_{n-1} + \cdots + a_n B_0 \\
    & = a_0 (B + \beta_n) + a_1 (B + \beta_{n-1}) + \cdots + a_n (B + \beta_0) \\
    & = A_n B + a_0 \beta_n + a_1 \beta_{n-1} + \cdots a_n \beta_0.
\end{align*}
Let
\begin{align*}
    \gamma_n = a_0 \beta_n + a_1 \beta_{n-1} + \cdots a_n \beta_0,
\end{align*}
and it suffices to show that $\lim_{n\to\infty} \gamma_n = 0$, since $A_nB \to AB$. Since $\sum a_n$ converges absolutely, let
\begin{align*}
    \alpha = \sum^\infty_{n=0} \left|a_n\right|.
\end{align*}
For any $\varepsilon > 0$, since $\sum b_n$ converges, $\beta_n \to 0$, then there is an integer $N > 0$ such that for all $n \geq N$, $\left|\beta_n\right| < \varepsilon$, and hence
\begin{align*}
    \left|\gamma_n\right| & \leq \left|a_n \beta_0 + \cdots + a_{n-N} \beta_N\right| + \left|a_{n-N+1} \beta_{N+1} + \cdots + a_0 \beta_n\right| \\
    & \leq \left|a_n \beta_0 + \cdots + a_{n-N} \beta_N\right| + \alpha \varepsilon.
\end{align*}
Keeping $N$ fixed and letting $n \to \infty$ implies
\begin{align*}
    \limsup_{n\to\infty} \left|\gamma_n\right| \leq \alpha \varepsilon,
\end{align*}
since $a_k \to 0$ as $k \to \infty$. Since $\varepsilon$ is arbitrary, $\lim_{n\to\infty} \gamma_n = 0$.
\end{proof}

\medskip

\begin{definition}
Let $\sum^\infty_{n=1} a_n$ be a series and let $\phi: \mathbb{N} \to \mathbb{N}$, then a series $\sum^\infty_{n=1} a_{\phi(n)}$ is obtained from $\sum^\infty_{n=1} a_n$ by rearrangement of the elements. 
\end{definition}

\medskip

\begin{theorem}
If $\sum^\infty_{n=1} a_n$ converges absolutely and $\phi: \mathbb{N} \to \mathbb{N}$ is a bijection, then $\sum^\infty_{n=1} a_{\phi(n)}$ converges and 
\begin{align*}
    \sum^\infty_{n=1} a_{\phi(n)} = \sum^\infty_{n=1} a_n.
\end{align*}
\end{theorem}
\begin{proof}
By Cauchy criterion, for every $\varepsilon > 0$, there is an integer $N > 0$ such that for all $n \geq N$,
\begin{align*}
    \left|a_N\right| + \left|a_{N+1}\right| + \cdots + \left|a_n\right| < \varepsilon.
\end{align*}
Let $A_n = \sum^n_{k=1} a_k$ and $R_n = \sum^n_{k=1} a_{\phi(k)}$. Chooses $p$ so large such that
\begin{align*}
    \{1,2,\cdots,N-1\} \subset \{\phi(1),\phi(2),\cdots,\phi(p)\}.
\end{align*}
If $n > p$, then the numbers $a_1,a_2, \cdots, a_{N-1}$ will cancel out in the difference of the partial sums
\begin{align*}
    A_n - R_n = \sum^n_{k=1} a_k - \sum^n_{k=1} a_{\phi(k)}.
\end{align*}
And the remaining terms will be $a_i$ with $i \geq N$ and the signs are $+$ or $-$. Hence there is an integer $m > N$ such that 
\begin{align*}
    \left|A_n - R_n\right| \leq \sum^m_{k=N} \left|a_k\right| < \varepsilon.
\end{align*}
We proved that for every $\varepsilon > 0$, there is an integer $N > 0$ such that for all $n \geq N$, $\left|A_n - R_n\right| < \varepsilon$, which implies that $R_n$ converges to the same limit as $A_n$.
\end{proof}

\medskip

\begin{example}
Consider the series
\begin{align*}
    1 - \frac{1}{2} + \frac{1}{3} - \frac{1}{4} + \frac{1}{5} + \cdots.
\end{align*}
The series is convergent conditionally, but not absolutely. Let $\{s_n\}$ be the sequence of its partial sums. It follows that
\begin{align*}
    \frac{5}{6} = 1 - \frac{1}{2} + \frac{1}{3} = s_3 > s_5 > s_7 > \cdots.
\end{align*}
Now we change the order of elements as follows
\begin{align*}
    \underbrace{1 + \frac{1}{3} - \frac{1}{2}}_{>\,0} + \underbrace{\frac{1}{5} + \frac{1}{7} - \frac{1}{4}}_{>\,0} + \underbrace{\frac{1}{9} + \frac{1}{11} - \frac{1}{6}}_{>\,0} + \cdots.
\end{align*}
and clearly the series cannot converges to $5/6$.
\end{example}

\medskip

\begin{theorem}[Riemann]
If a series $\sum^\infty_{n=1} a_n$ converges conditionally, but not absolutely, then for every $g \in \overline{\mathbb{R}}$, there is a bijection $\phi: \mathbb{N} \to \markright{N}$ such that
\begin{align*}
    \sum^\infty_{n=1} a_{\phi(n)} = g.
\end{align*}
\end{theorem}




\chapter{Continuity}


\section{Limits of Functions}

\begin{definition}
Let $X$ and $Y$ be metric spaces, suppose $E \subset X$, $f$ maps $E$ into $Y$, and $x_0$ is a limit point of $E$. We write $f(x) \to y_0$ as $x \to x_0$, or
\begin{align*}
    \lim_{x \to x_0} f(x) = y_0,
\end{align*}
if for every $\varepsilon > 0$, there exists a $\delta > 0$ such that $d_Y(f(x), y_0) < \varepsilon$ for all points $x \in E$ for which $0 < d_X(x,x_0) < \delta$.

The symbols $d_X$ and $d_Y$ refer to the distance in $X$ and $Y$, respectively. 
\end{definition}

\begin{remark}
Note that $x_0 \in X$, but $x_0$ need not be a point of $E$. 
\end{remark}

\medskip

\begin{theorem}\label{th_31}
Let $X$, $Y$, $E$, $f$ and $x_0$ be as in the definition above. Then $\lim_{x \to x_0} f(x) = y_0$ if and only if $\lim_{n \to \infty} f(x_n) = y_0$ for every sequence $\{x_n\}$ in $E$ such that $\lim_{n\to\infty} x_n = x_0$ and $x_n \neq x$.
\end{theorem}
\begin{proof}
Suppose $\lim_{x \to x_0} f(x) = y_0$. Let $\{x_n\}$ be a sequence in $E$ such that $x_n \neq x$ and $\lim_{n\to\infty} x_n = x_0$. For every $\varepsilon > 0$, there exists $\delta > 0$ such that $d_Y(f(x),y_0) < \varepsilon$ for all $x \in E$ with $d_X(x,x_0) < \delta$. Also, there exists an integer $N > 0$ such that for all $n \geq N$, $d_X(x_n,x_0) < \delta$. Hence, for all $n \geq N$, we have $d_Y(x_n,x_0) < \varepsilon$, which implies $\lim_{n \to \infty} f(x_n) = y_0$.

Conversely, suppose $\lim_{x \to x_0} f(x) \neq y_0$. Then there exists some $\varepsilon$ such that for every $\delta > 0$, there is a point $x \in E$ for which $d_X(x,x_0) < \delta$ but $d_Y(f(x),y_0) > \varepsilon$. Taking $\delta = 1/n, n = 1,2,3,\cdots$, we obtain a sequence $\{x_n\}$ in $E$ such that $\lim_{n\to\infty} x_n = x_0$ but $\lim_{n\to\infty} f(x_n) \neq y_0$, which is a contradiction.
\end{proof}

\medskip

\begin{theorem}
If $f$ has a limit at $x_0$, this limit is unique. 
\end{theorem}
\begin{proof}
It follows from Theorem \ref{th_21} \ref{th_21_b} and Theorem \ref{th_31}.
\end{proof}

\medskip

\begin{theorem}
Suppose $E \subset X$, a metric space, $x_0$ is a limit point of $E$. $f$ and $g$ are complex functions on $E$, and
\begin{align*}
    \lim_{x\to x_0} f(x) = A, \quad \lim_{x\to x_0} g(x) = B.
\end{align*}
Then,
\begin{enumerate}[label=(\alph*)]
    \item $\lim_{x\to x_0} (f + g)(x) = A + B$;
    
    \item $\lim_{x\to x_0} (fg)(x) = AB$;
    
    \item $\lim_{x\to x_0} \displaystyle \left(\frac{f}{g}\right)(x) = \frac{A}{B}$, if $B \neq 0$.
\end{enumerate}
\end{theorem}
\begin{proof}
The proof is obvious.
\end{proof}



\medskip



\section{Continuous Functions}

\begin{definition}
Suppose $X$ and $Y$ are metric spaces, $E \subset X$ and $x_0 \in E$, and $f: E \to Y$. Then $f$ is said to be continuous at $x_0$ if for every $\varepsilon > 0$, there exists a $\delta > 0$ such that 
\begin{align*}
    d_Y(f(x),f(x_0)) < \varepsilon,
\end{align*}
for all points $x \in E$ for which $d_X(x,x_0) < \delta$. 

If $f$ is continuous at every point of $E$, then $f$ is said to be continuous on $E$.
\end{definition}

\begin{remark}
If $x_0$ is an isolated point of $E$, then the definition implies that every function $f$ which has $E$ as its domain is continuous at $x_0$. Indeed, for every $\varepsilon > 0$, we can choose $\delta > 0$ os that the only point $x \in E$ for which $d_X(x,x_0) < \delta$ is $x = x_0$, then $d_Y(f(x),f(x_0)) = 0 < \varepsilon$.
\end{remark}

\medskip

\begin{theorem}
With the same assumption in the above definition, and also assume that $x_0$ is a limit point of $E$. Then $f$ is continuous at $x_0$ if and only if $\lim_{x\to x_0} f(x) = f(x_0)$.
\end{theorem}

\medskip

\begin{theorem}
Suppose $X, Y, Z$ are metric spaces, $E \subset X$, $f: E \to Y$ and $g: f(E) \to Z$, and $h$ is the mapping of $E$ into $Z$ defined by
\begin{align*}
    h(x) = g(f(x)).
\end{align*}
If $f$ is continuous at a point $x_0 \in E$ and if $g$ is continuous at the point $f(x_0)$, then $h$ is continuous at $x_0$.
\end{theorem}
\begin{proof}
For every $\varepsilon > 0$, since $g$ is continuous at $f(x_0)$, there exists $\eta > 0$ such that 
\begin{align*}
    d_Z(g(y),g(x_0)) < \varepsilon,
\end{align*}
if $d_Y(y,f(x_0)) < \eta$ for $y \in f(E)$. Since $f$ is continuous at $x_0$, there exists a $\delta > 0$ such that
\begin{align*}
    d_Y(f(x),f(x_0)) < \eta,
\end{align*}
if $d_X(x,x_0) < \delta$ for $x \in E$. It follows that
\begin{align*}
    d_Z(h(x),h(x_0)) = d_Z(g(f(x)), g(f(x_0))) < \varepsilon,
\end{align*}
if $d_X(x,x_0) < \delta$ for $x \in E$. Thus $h$ is continuous at $x_0$.
\end{proof}

\medskip

\begin{theorem}\label{th_36}
A mapping $f$ of a metric space $X$ into a metric space $Y$ is continuous if and only if $f^{-1}(V)$ is open in $X$ for every open set $V$ in $Y$.
\end{theorem}
\begin{proof}
Suppose $f$ is continuous on $X$ and $V$ is an open set in $Y$. We need to show that every point of $f^{-1}(V)$ is an interior point of $f^{-1}(V)$. So suppose $x \in X$, and $f(x) \in V$. Since $V$ is open, there exists $\varepsilon > 0$ such that for all points $y$ with $d_Y(f(x),y) < \varepsilon$, $y \in V$. Since $f$ is continuous, then there exists a $\delta > 0$ such that $d_Y(f(x), f(x')) < \varepsilon$ if $d_X(x,x') < \delta$. Hence $x' \in f^{-1}(V)$ if $d_X(x,x') < \delta$.

Conversely, suppose $f^{-1}(V)$ is open in $X$ for every open set $V$ in $Y$. For $x \in X$ and every $\varepsilon > 0$, let $V$ be the set of all $y \in Y$ such that $d_Y(y,f(x)) < \varepsilon$. Clearly $V$ is open, hence $f^{-1}(V)$ is open, hence there exits $\delta > 0$ such that $x' \in f^{-1}(V)$ if $d_X(x,x') < \delta$. But if $x' \in f^{-1}(V)$, then $f(x') \in V$, so $d_Y(f(x), f(x')) < \varepsilon$.
\end{proof} 

\medskip

\begin{corollary}
A mapping $f$ of a metric space $X$ into a metric space $Y$ is continuous if and only if $f^{-1}(V)$ is closed in $X$ for every closed set $V$ in $Y$.
\end{corollary}
\begin{proof}
If follows from the theorem, since a set is closed if and only if its complement is open, and since $f^{-1}(E^c) = \left(f^{-1}(E)\right)^c$ for every $E \subset X$.
\end{proof}

\medskip

\begin{theorem}\label{th_37}
Let $f$ and $g$ be complex continuous functions on a metric space $X$. Then $f + g$, $fg$ and $f/g$ are continuous on $X$, when $g(x) \neq 0$ for all $x \in X$.
\end{theorem}

\medskip

\begin{theorem}\label{th_38}
~\begin{enumerate}[label=(\alph*)]
    \item Let $f_1, \cdots, f_n$ be real functions on a metric space $X$, and let $f: X \to \mathbb{R}^k$ be defined by
    \begin{align*}
        f(x) = \left(f_1(x), \cdots, f_n(x)\right),
    \end{align*}
    then $f$ is continuous if and only if each of the functions $f_1, \cdots, f_n$ is continuous. \label{th_38_a}
    
    \item If $f,g: X \to \mathbb{R}^n$ are continuous mappings, then $f + g$ and $fg$ are continuous. \label{th_38_b}
\end{enumerate}
\end{theorem}
\begin{proof}
Part \ref{th_38_a} follows from the inequalities
\begin{align*}
    \left|f_i(x) - f_i(y)\right| \leq \left|f(x) - f(y)\right| = \left(\sum^n_{i=1} \left|f_i(x) - f_i(y)\right|^2\right)^{1/2},
\end{align*}
for $i = 1,2,\cdots,n$. Part \ref{th_38_b} follows from \ref{th_38_a} and Theorem \ref{th_37}.
\end{proof}

\medskip




\section{Continuity and Compactness}

\begin{definition}
Suppose $E \subset X$. A mapping $f: E \to \mathbb{R}^n$ is said to be bounded if there is a real number $M$ such that $\left|f(x)\right| \leq M$ for all $x \in E$. 
\end{definition}

\medskip

\begin{theorem}\label{th_39}
Suppose $f$ is a continuous mapping of a compact metric space $X$ into a metric space $Y$, then $f(X)$ is compact.
\end{theorem}
\begin{proof}
Let $\{V_{\alpha}\}$ be an open cover of $f(X)$. Since $f$ is continuous, by Theorem \ref{th_36}, each $f^{-1}(V_{\alpha})$ is open. Since $X$ is compact, there is a finite subcoverings $\{V_{\alpha_1}, \cdots, V_{\alpha_n}\}$ such that
\begin{align*}
    X \subset \bigcup^n_{i=1} f^{-1}(V_{\alpha_i}).
\end{align*}
Since $f\left(f^{-1}(E)\right) \subset E$ for every $E \subset Y$, hence we have
\begin{align*}
    f(X) \subset \bigcup^n_{i=1} V_{\alpha_i},
\end{align*}
and hence there is a finite subcoverings that covers $f(X)$, which implies $f(X)$ is compact.
\end{proof}

\medskip

\begin{theorem}\label{th_310}
If $f$ is a continuous mapping of a compact metric space $X$ into $\mathbb{R}^n$, then $f(X)$ is closed and bounded. Thus, $f$ is bounded.
\end{theorem}
\begin{proof}
It follows from Theorem \ref{th_123}.
\end{proof}

\begin{remark}
This result is important when $f$ is a real function.
\end{remark}

\medskip

\begin{theorem}\label{th_311}
Suppose $f$ is a continuous real function on a compact metric space $X$, and
\begin{align*}
    M = \sup_{x\in X} f(x), \quad m = \inf_{x\in X} f(x).
\end{align*}
Then there exist points $x_1, x_2 \in X$ such that $f(x_1) = M$ and $f(x_2) = m$.
\end{theorem}
\begin{proof}
By Theorem \ref{th_310}, $f(X)$ is a bounded and closed subset of $\mathbb{R}$, hence $f(X)$ contains $M$ and $m$ by Theorem \ref{th_113}.
\end{proof}

\begin{remark}
This theorem can also be stated as follows: There exists $x_1, x_2 \in X$ such that $f(x_2) \leq f(x) \leq f(x_1)$ for all $x \in X$, that is, $f$ attains its maximum and minimum. Now we provide another proof to this theorem.
\end{remark}

\medskip

\begin{proof}[Second Proof of Theorem \ref{th_311}]
clearly there is a sequence $\{x_n\} \in X$ such that $\lim_{n\to\infty} f(x_n) = \sup_{x\in X} f(x)$ (even if the supremum equals $\infty$).  Since the sequence $\{x_n\}$ is bounded, then it has a convergent subsequence $\{x_{n_k}\}$ by Theorem \ref{th_210}, for which $x_{n_k} \to x_1 \in X$ and hence by the continuity of $f$,
\begin{align*}
    \sup_{x \in X} f(x) = \lim_{n\to\infty} f(x_{n_k}) = f(x_1).
\end{align*}
Similarly, there is $x_2 \in X$ such that $f(x_2) = \inf_{x\in X} f(x)$. 
\end{proof}

\medskip

\begin{theorem}
Suppose $f$ is a continuous one-to-one mapping of a compact metric space $X$ onto a metric space $Y$. Then the inverse mapping $f^{-1}$ defined on $Y$ given by
\begin{align*}
    f^{-1}(f(x)) = x, \,\, x \in X,
\end{align*}
is a continuous mapping of $Y$ onto $X$.
\end{theorem}
\begin{proof}
Applying Theorem \ref{th_36} to $f^{-1}$ rather than $f$, and it suffices to show that $f(V)$ is open whenever $V$ is open on $X$. Fix such a set $V$.

The complement $V^c$ of $V$ is closed in $X$, hence compact by Theorem \ref{th_116}, hence $f(V^c)$ is a compact subset of $Y$ by Theorem \ref{th_39}. Since $f$ is one-to-one and onto, $f(V)$ is the complement of $f(V^c)$. Hence $f(V)$ is open.
\end{proof}

\medskip


\section{Uniform Continuity}

\begin{definition}
Let $f: X \to Y$. We say that $f$ is uniformly continuous on $X$ if for every $\varepsilon > 0$ there exists $\delta > 0$ such that 
\begin{align*}
    d_Y(f(x),f(y)) < \varepsilon,
\end{align*}
for all $x,y \in X$ for which $d_X(x,y) < \delta$.
\end{definition}

\begin{remark}
The continuity and uniformly continuity are equivalent on compact sets, which will be shown in the next theorem.
\end{remark}

\medskip

\begin{theorem}\label{th_313}
Let $f$ be a continuous mapping of a compact metric space $X$ into a metric space $Y$. Then, $f$ is uniformly continuous.
\end{theorem}
\begin{proof}
Let $\varepsilon > 0$ be given. Since $f$ is continuous, then for each $x \in X$, there is $\phi(x)$ such that
\begin{align}\label{th_313_equ1}
    d_Y(f(x),f(y)) < \frac{\varepsilon}{2},
\end{align}
for all $y \in X$ with $d_X(x,y) < \phi(x)$. Let $B_x = \{y \in X \,:\, d_X(x,y) < \phi(x)/2\}$. Clearly $\{B_x\}$ is an open cover of $X$, and since $X$ is compact, there is a finite subcovering $\{B_{x_1}, \cdots, B_{x_n}\}$ such that 
\begin{align*}
    X \subset \bigcup^n_{i=1} B_{x_i}.
\end{align*}
Let $\delta = \min\{\phi(x_1), \cdots, \phi(x_n)\}/2$. Now for any $x,y \in X$ such that $d_X(x,y) < \delta$. Also, there is $j \in \{1,\cdots,n\}$ such that $x \in B_{x_j}$, hence $d_X(x,x_j) < \phi(x_j)/2$. And we have
\begin{align*}
    d_X(y, x_j) \leq d_X(y,x) + d_X(x, x_j) < \delta + \frac{\phi(x_j)}{2} \leq \phi(x_j).
\end{align*}
Finally, \eqref{th_313_equ1} implies
\begin{align*}
    d_Y(f(x),f(y)) \leq d_Y(f(x), f(x_j)) + d_Y(f(x_j), f(y)) < \varepsilon,
\end{align*}
and thus $f$ is uniformly continuous.
\end{proof}



\begin{proof}[Second Proof of Theorem \ref{th_313}]
Suppose to the contrary that $f$ is not uniformly continuous, that is, there exists $\varepsilon > 0$ such that for all $\delta > 0$, there are $x,y \in X$ such that $d_Y(f(x),f(y)) \geq \varepsilon$ if $d_X(x,y) < \delta$. 

Taking $\delta = 1/n$, then there exist sequences $\{x_n\}$ and $\{y_n\}$ such that $d_Y(f(x_n),f(y_n)) \geq \varepsilon$ if $d_X(x_n,y_n) < 1/n$. Since $X$ is compact, then $\{x_n\}$ has a subsequence $\{x_{n_k}\}$ such that $x_{n_k} \to x_0 \in X$. Since $d_X(x_{n_k},y_{n_k}) < 1/n_k$, $y_{n_k}$ also converges to $x_0$. By the continuity of $f$, we have 
\begin{align*}
    \lim_{k\to\infty} f(x_{n_k}) = f(x_0), \quad \lim_{k\to\infty} f(y_{n_k}) = f(x_0),
\end{align*}
and this contradicts with $_Y(f(x_{n_k}),f(y_{n_k})) \geq \varepsilon$.
\end{proof}

\medskip

\begin{theorem}\label{th_314}
Let $E$ be a noncompact set in $\mathbb{R}$. Then,
\begin{enumerate}[label=(\alph*)]
    \item there exists a continuous function on $E$ which is not bounded; \label{th_314_a}
    
    \item there exists a continuous and bounded function on $E$ which has no maximum. \label{th_314_b}
\end{enumerate}
If, in addition, $E$ is bounded, then
\begin{enumerate}[label=(\alph*)]
    \setcounter{enumi}{2}
    \item there exists a continuous function on $E$ which is not uniformly continuous. \label{th_314_c}
\end{enumerate}
\end{theorem}
\begin{proof}
Suppose first that $E$ is bounded, so that there exists a limit point $x_0$ of $E$ which is not a point of $E$. Consider
\begin{align*}
    f(x) = \frac{1}{x - x_0}, \,\, x \in E.
\end{align*}
Clearly $f$ is continuous but not unbounded. Also, $f$ is not uniformly continuous. Now consider 
\begin{align*}
    g(x) = \frac{1}{1 + (x - x_0)^2}, \,\, x \in E.
\end{align*}
Hence $g$ is continuous on $E$ and is bounded, since $0 < g(x) < 1$. It is clear that
\begin{align*}
    \sup_{x\in E} g(x) = 1,
\end{align*}
and thus $g$ has no maximum on $E$. 

Now suppose $E$ is unbounded. Then $f(x) = x$ proves part \ref{th_314_a}, and 
\begin{align*}
    h(x) = \frac{x^2}{1 + x^2}, \,\, x \in E,
\end{align*}
proves part \ref{th_314_b}. For part \ref{th_314_c}, let $E = \mathbb{Q} \cap [0,2]$, and clearly $E$ is bounded. Now define $f$ on $E$ as
\begin{align*}
    l(x) = \begin{cases}
        0, & 0 \leq x < \sqrt{2}, \\
        1, & \sqrt{2} < x \leq 2.
    \end{cases}
\end{align*}
Clearly $l$ is continuous on $E$, since $f^{-1}(\{0\})$ and $f^{-1}(\{1\})$ are closed in $\mathbb{Q}$. However, $f$ is not uniformly continuous.
\end{proof}














\newpage
\bibliographystyle{unsrt}
\bibliography{bibliography}

\end{document}